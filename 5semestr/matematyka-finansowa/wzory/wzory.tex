\documentclass{article}

\begin{document}
\begin{center}

\section{Kapitalizacja prosta}

	\begin{equation}
		I_n = Wnr
	\end{equation}
	\begin{equation}
		W_n = W(1+nr)
	\end{equation}
	\begin{equation}
		W_n = W(1 + \Sigma^m_{i=0}r_in_i)
	\end{equation}
	\begin{equation}
		I_n = W\Sigma^m_{i=0}r_in_i
	\end{equation}
	\begin{equation}
		r = \frac{1}{n}\Sigma^m_{i=1}r_in_i
	\end{equation}
	\begin{equation}
		r = \frac{1}{m}\Sigma^m_{i=1}r_i
	\end{equation}
	\begin{equation}
		I_k = Wkr_{okr}
	\end{equation}
	\begin{equation}
		W_k = W(1 + kr_{okr})
	\end{equation}
	\begin{equation}
		m_1 * r_1 = m_2 * r_2
	\end{equation}
	
	\section{Kapitalizacja złożona}
	
	\begin{equation}
		W_n = W(1+r)^n
	\end{equation}
	\begin{equation}
		I_n = W((1+r)^n - 1)
	\end{equation}
	\begin{equation}
		W_n = W\Pi^n_{i=1}(1 + r_i)
	\end{equation}
	\begin{equation}
		I_n = W(\Pi^n_{i=1}(1 + r_i) - 1)
	\end{equation}
	\begin{equation}
		r = (\Pi^n_{i=1}(1 + r_1))^{\frac{1}{n}} - 1
	\end{equation}
	\begin{equation}
		W^{(m)}_{(l, n)} = W(1 + r_{okr})^{l * m + n}
	\end{equation}
	\begin{equation}
		W^{(m)}_{(l, n)} = W(1 + \frac{r}{m})^{l*m+n}
	\end{equation}
	\begin{equation}
		W^{(m)}_{l} = W(1 + \frac{r}{m})^{l*m}
	\end{equation}
	\begin{equation}
		R_m = (1 + \frac{r}{m})^m
	\end{equation}
	
	\newpage
	
	\section{Równoważność stóp pod okresowych przy kapitalizacji złożonej}
	
	\begin{equation}
		(1 + r_1)^{m_1} = (1 + r_2)^{m_2}
	\end{equation}
	\begin{equation}
		(1 + \frac{r_1}{m_1})^{m_1} = (1 + \frac{r_2}{m_2})^{m_2}
	\end{equation}
	
	\section{Efektywna stopa procentowa}
	
	\begin{equation}
		r^{(m)}_{ef} = (1 + r_{okr})^m - 1
	\end{equation}
	\begin{equation}
		r^{(m)}_{ef} = (1 + \frac{r}{m})^m - 1
	\end{equation}
	
	\section{Kapitalizacja ciągła}
	
	\begin{equation}
		W^{(\infty)}_l = We^{l * r}
	\end{equation}
	
	\section{Natężenie procentowe}
	
	\begin{equation}
		l + r_{ef} = e^r
	\end{equation}
	\begin{equation}
		r_{ef} = e^r - 1
	\end{equation}
	\begin{equation}
		r = ln(1 + r_{ef})
	\end{equation}
	
	\section{Dyskonto proste i składane}
	
	\begin{equation}
		PV(W) = \frac{W}{1 + nr}
	\end{equation}
	\begin{equation}
		D = \frac{nrW}{1 + nr}
	\end{equation}
	\begin{equation}
		PV(W) = \frac{W}{(1 + r)^n}
	\end{equation}
	\begin{equation}
		PD = W(1 - \frac{1}{(1 + r)^n})
	\end{equation}
	
	\section{Dyskonto przy wielokrotnej kapitalizacji w ciągu roku}
	
	\begin{equation}
		PV(W) - \frac{W}{(1 + \frac{r}{m})^{lm + n}}
	\end{equation}
	\begin{equation}
		D = W(1 - \frac{1}{(1 + \frac{r}{m})^{lm + n}})
	\end{equation}
	\begin{equation}
		PV(W) - \frac{W}{(1 + \frac{r}{m})^{lm}}
	\end{equation}
	\begin{equation}
		D = W(1 - \frac{1}{(1 + \frac{r}{m})^{lm}})
	\end{equation}
	
	\section{Dyskonto przy kapitalizacji ciągłej}
	
	\begin{equation}
		PV(W) = W * e^{-r*n}
	\end{equation}
	\begin{equation}
		D = W(1 = e^{-r * n})
	\end{equation}
	
	\section{Dyskonto handlowe}
	
	\begin{equation}
		D = F - P
	\end{equation}
	
	\section{Stopa dyskontowa}
	
	\begin{equation}
		d = \frac{D - P}{nF}
	\end{equation}
	
	
	
\end{document}