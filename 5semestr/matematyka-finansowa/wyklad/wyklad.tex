\documentclass{article}

\begin{document}

\tableofcontents

\newpage

%
% Sekcja 1
%

\section{Wstęp}

\textbf{Odestkami} - nazywa się kwotę, którą należy zapłacić za prawo użytkowania określonego kapitału. Odsetki są zatem ceną płaconą za wypożyczenie kapitału. Ustala się je w odniesieniu do pewnego ustalonego okresu. Stosunek odsetek do kapitału, który je wygenerował w ustalonym okresie, nazywa się \textbf{okresową stopą procentową.}\\

W praktyce najczęściej mamy do czynienia ze stopami procentowymi ustalonymi dla okresy rocznego. Mówimy wtedy o \textbf{rocznej stopie procentowej}.\\

Jeżeli np. odsetki za 1 rok od pożyczonego kapitału 60 000 PLN wynoszą 1 500 PLN, to roczna stopa procentowa jest równa $ r = \frac{1 500}{60 000} = 2,5 \% $.\\

Powiększenie kapitału o odsetki, które zostały przez niego wygenerowane, nazywa się \textbf{kapitalizacją odsetek}. Czas, w którym odsetki są generowane, nazywa się okresem kapitalizacji. W dalszym ciągu rozważań ograniczymy się do przypadku, gdy odsetki są dopisywane na końcu okresów kapitalizacji. Mówimy wtedy o kapitalizacji z dołu.\\

Wyróżniamy dwa podstawowe rodzaje kapitalizacji: \textbf{prostą i złożoną}.\\

%
% Sekcja 2
%

\section{Kapitalizacja prosta}

W przypadku kapitalizacji prostej odsetki od kapitału oblicza się od kapitału początkowego proporcjonalnie od długości kresu oprocentowania. Oznaczamy przez $ W $ początkową wartość kapitału, przez $ r $ roczną stopę procentową, przez $ I_{n} $ należne za czas $ n $, zaś przez $ W_{n} $ oznaczamy końcową wartość kapitału w czasie $ n $ (w latach).\\

\textbf{Reguła bankowa} – każdy rok ma 360 dni, zaś każdy miesiąc ma 30 dni.

\begin{center}
	\begin{equation}
		I_n = Wnr
	\end{equation}
\end{center}\\

Natomiast wartość końcowa kapitału:

\begin{center}
	\begin{equation}
		W_n = W(1+nr)
	\end{equation}
\end{center}\\

\newpage

\subsection{Przykład 1}
Przy kapitalizacji prostej i rocznej stopie procentowej $ r = 4\% $ wyznaczyć odsetki i końcową wartość kapitału 25 000 PLN po upływie a) 3lat, b) 142dni.\\

\subsubsection{a)}

$ I_n = 25 000 * 3 * 0,04 = 3 000 PLN $

\subsubsection{b)}

$ W_n = 25 000(1 + \frac{142}{360} + 0.04) = 25 394,44 PLN $\\

Załóżmy że czas trwania inwestycji wynosi $ n $ lat i składa się z $ m $ następujących po sobie okresów o długości $ n_1, ...., n_m $. Przyjmijmy że w każdym z nich obowiązuje roczna stopa procentowa, odpowiednio, $ r_1, ..., r_m $. Wtedy wartość kapitału początkowego $ W $ po pierwszym okresie wyniesie: 

\begin{center}
	\begin{equation}
		W_n = W(1 + \Sigma^m_{i=0}r_in_i)
	\end{equation}
\end{center}

\begin{center}
	\begin{equation}
		I_n = W\Sigma^m_{i=0}r_in_i
	\end{equation}
\end{center}\\

\textbf{Przeciętną roczną stopą procentową oprocentowania kapitału} $ W $ w czasie $ n $ nazywa się roczną stopę, przy której kapitał $ W $ generuje w czasie $ n $ odsetki o takiej samej wartości jak przy stopach zmiennych. Definicja ta dotyczy zarówno kapitalizacji prostej i złożonej.\\

Oznaczając przez r(z kreską na górze) przeciętną roczną stopą oprocen, na podstawie wzorów (1) i (4) mamy

\begin{center}
	\begin{equation}
		r = \frac{1}{n}\Sigma^m_{i=1}r_in_i
	\end{equation}
\end{center}\\

Gdyby wszystkie okresy miały jednakową długość to wzór:

\begin{center}
	\begin{equation}
		r = \frac{1}{m}\Sigma^m_{i=1}r_i
	\end{equation}
\end{center}\\

\newpage

\subsection{Przykład 2}
Przez początkowe 4 miesiąca trwania obowiązywała roczna stopa procentowa 6%, przez następnych 5 miesięcy 7%, przez ostatnie 3 miesiące 7,5%.\\

Dane:\\

$ N_1 = \frac{4}{12} $\\

$ N_2 = \frac{5}{12} $\\

$ N_3 = \frac{3}{12} $\\

$ R_1 = 0,06 $\\

$ R_2 = 0,07 $\\

$ R_3 = 0,075 $\\

$ W = 20 000 PLN $

\subsubsection{a)}
Korzystając ze wzoru (3) mamy\\
$ W_3 = 20000(1 + 0.06*\frac{4}{12} + 0.07*\frac{5}{12} + 0.075*\frac{3}{12}) = 21 358,40 PLN $

\subsubsection{b)}
Obliczyć wysokość przeciętnej rocznej stopy oprocentowania\\
Korzystając ze wzory (5) mamy\\
$ r =  0.06*\frac{4}{12} + 0.07*\frac{5}{12} + 0.075*\frac{3}{12} = 6,79\% $\\

Często zdarza się, że stopa procentowa, przy której należy obliczyć odsetki nie jest stopa roczną lecz np. miesięczną lub kwartalną. Okres, po którym odsetki podlegają kapitalizacji nazywa się \textbf{podokresem kapitalizacji}. Stopa procentowa ustalona dla podokresu kapitalizacji nazywa się \textbf{stopą pod okresową. Częstotliwość kapitalizacji} oznacza ile razy odsetki są kapitalizowane w ciągu roku.\\

W dalszym ciągu zakładamy że częstotliwość kapitalizacji wynosi $ m $. Wobec tego każdy rok jest podzielony na $ m $ równych podokresów kapitalizacji.\\

$ m=1 $ – kapitalizacja roczna\\

$ m=2 $ – kapitalizacja półroczna\\

$ m=4 $ – kapitalizacja kwartalna\\

$ m=12 $ – kapitalizacja miesięczna\\

$ m=360 $ – kapitalizacja dobowa(dzienna)\\

Jeżeli $ r_{okr} $ jest stopą podokresową, to zgodnie z zasadą oprocentowania prostego odsetki od kapitału $ W $ po upływie $ k $ podokresów wyznacza się ze wzoru

\begin{center}
	\begin{equation}
		I_k = Wkr_{okr}
	\end{equation}
\end{center}\\

Natomiast końcowa wartość kapitału W po upływie $ k $:\\

\begin{center}
	\begin{equation}
		W_k = W(1 + kr_{okr})
	\end{equation}
\end{center}\\

Załóżmy że $ r_1 $ i $ r_2 $ są podokresowymi stopami procentowymi, zaś $ m_1 $ i $ m_2 $ są odpowiadającymi im częstotliwościami kapitalizacji. Stopy $ r_1 $ i $ r_2 $ nazywamy równoważnymi w czasie $ n $, jeżeli przy każdej z nich odsetki od ustalonego kapitału po czasie $ n $ są równe.\\

Korzystając z (7) mamy:\\

\begin{center}
	\begin{equation}
		m_1 * r_1 = m_2 * r_2
	\end{equation}
\end{center}\\

Z (9) stopy pod okresowe są wtedy i tylko wtedy ich stosunek jest równy stosunkowi długości odpowiadających im po okresów. Takie stopy pod okresowe nazywają się \textbf{proporcjonalnymi}.

\subsection{Przykład 3}
Kwartalna stopa oprocentowania prostego wynosi 6%. Wyznaczyć następujące równoważne stopy oprocentowania prostego: 

\subsubsection{a) roczna}
$ 6 * 4 = 24\% $
\subsubsection{b) miesięczna}
$ 6 / 3 = 2\% $
\subsubsection{c) tygodniowa}
$ 6 / 12 = 0,5\% $

\newpage

%
% Sekcja 3
%

\section{Kapitalizacja złożona}
W przypadku \textbf{kapitalizacji złożonej} odsetki oblicza się za każdy okres równy okresowi kapitalizacji i kapitalizuje się je na koniec tego okresu. Załóżmy, że kwota $ W $ została ulokowana na rachunku z roczną stopą procentową równą $ r $. W przypadku kapitalizacji złożonej dochód przynosi początkowy kapitał wraz z odsetkami uzyskanymi na koniec poprzedniego okresu kapitalizacji. Przez $ I_n $ oznaczmy odsetki należne po czasie $ n $, zaś przez $ W_n $ oznaczmy wartość kapitału po $ n $ latach. Wtedy: $ W_1 = w(1+r) $\\

\begin{center}
	\begin{equation}
		W_n = W(1+r)^n
	\end{equation}
\end{center}\\

Liczba $ (1+r)^n $ nazywa się \textbf{czynnikiem wartości przyszłej} w kapitalizacji złożonej.\\

Odsetki po okresie n lat wynoszą:\\

\begin{center}
	\begin{equation}
		I_n = W((1+r)^n - 1)
	\end{equation}
\end{center}\\

\subsection{Przykład 4}
Przy założeniu kapitalizacji złożonej i rocznej stopie procentowej r = 5\%, wyznaczymy wartość kapitału 40 000 PLN i odsetki po upływie 4 lat.\\

$ W_n = 40 000(1+ 0,05)^4 = 48 620 PLN $\\

$ I_n = 48 620 - 40 000 - 8 620 PLN $\\

$ I_n = 40 000((1 + 0,05)^4 - 1) = 8 620 PLN $\\

Podobnie jak w przypadku kapitalizacji prostej w kapitalizacji złożonej, możemy dopuścić zmienne stopy procentowe w kolejnych latach trwania inwestycji. Przyjmijmy, że w kolejnych latach stopy procentowe są równe $ r_1, r_2, ..., 4_n $ gdzie $ n $ jest liczą lat trwania inwestycji. Wtedy wartość początkowego kapitału $ W $ po pierwszym roku wyniesie. $ W_1 = W(1+r_1) $, po drugim $ W_2 = W(1+r_1)(1+r_2) $\\

Wartość kapitału po $ n $ latach: 

\begin{center}
	\begin{equation}
		W_n = W\Pi^n_{i=1}(1 + r_i)
	\end{equation}
\end{center}\\

\begin{center}
	\begin{equation}
		I_n = W(\Pi^n_{i=1}(1 + r_i) - 1)
	\end{equation}
\end{center}\\

Przeciętna roczna stopa oprocentowania w przypadku kapitalizacji złożonej:\\

\begin{center}
	\begin{equation}
		r = (\Pi^n_{i=1}(1 + r_1))^{\frac{1}{n}} - 1
	\end{equation}
\end{center}\\

\subsection{Przykład 5}
Kapitał 20 000 PLN został ulokowany na okres 5 lat. Przy założeniu kapitalizacji złożonej i rocznej stopie procentowej równej w kolejnych latach, 5\%, 6\%, 5\%, 4\%, 7\%, wyznaczymy wartości kapitału na koniec kolejnych lat oraz przeciętną roczną stopę oprocentowania tego kapitału w czasie 5 lat. \\

$ W_1 = 21 000 PLN $\\

$ W_5 = 20 000(1+0.05)(1+0.06)(1+0.05)(1+0.04)(1+0.07) = 26 009.47 PLN $ \\

$ r = ((1+0.05)(1+0.06)(1+0.05)(1+0.04)(1+0.07))^{\frac{1}{5}} – 1 = 5.40\% $\\

Niech $ r_{okr} $ będzie stopą pod okresową. Przy założeniu kapitalizacji złożonej, przyszła wartość kwoty $ W $ po l latach i $ n $ spośród $ m $ pod okresów $ l+1 $ roku, gdzie $ 0 \leq n < m $ wynosi:

\begin{center}
	\begin{equation}
		W^{(m)}_{(l, n)} = W(1 + r_{okr})^{l * m + n}
	\end{equation}
\end{center}\\

\subsection{Przykład 6}
Zakładając kapitalizację a) półroczną, b) kwartalną c) miesięczną i przyjmując stopę pod okresową $ r_{okr} $ = 2\% wyznaczyć przyszłą wartość kapitału 20 000 PLN po 2 latach i 6 miesiącach.

\subsubsection{a)}
$ W^{(2)}_{(2, 1)} = W(1 + 0,02)^{2 * 2 + 1} = 22 081,62 PLN $\\
\subsubsection{b)}
$ W^{(4)}_{(2, 2)} = W(1 + 0,02)^{10} = 24 379,89 PLN $\\
\subsubsection{c)}
$ W^{(12)}_{(2, 6)} = W(1 + 0,02)^{30} = 36 227,23 PLN $\\

Roczna stopa procentowa $ r $ proporcjonalna do danej stopy pod okresowej $ r_{okr} $ nazywa się \textbf{stopą nominalną}. ( wyliczyć roczną stopę, np. jak miesięczna jest 1\% to roczna jest 12\% itp.)

\begin{center}
	\begin{equation}
		W^{(m)}_{(l, n)} = W(1 + \frac{r}{m})^{l*m+n}
	\end{equation}
\end{center}\\

Przyjmując n = 0 wtedy:\\

\begin{center}
	\begin{equation}
		W^{(m)}_{l} = W(1 + \frac{r}{m})^{l*m}
	\end{equation}
\end{center}\\

Liczbę:\\

\begin{center}
	\begin{equation}
		R_m = (1 + \frac{r}{m})^m
	\end{equation}
\end{center}\\

Nazywa się rocznym \textbf{czynnikiem oprocentowania}.

\subsection{Przykład 7}
Kapitał w wysokości 40 000 PN został ulokowany na rachunku z nominalną stopą procentową równą 12\%. Zakładając kapitalizację, roczną, półroczną, kwartalną, miesięczną oraz dzienna, wyznaczyć przyszłą wartość kapitału po 4 latach.\\

Ze wzory (17)\\

$ W(1)4 = 62 940,77 PLN $\\ 

$ W(2)4 = 63 753.92 PLN $\\

$ W(4)4 = 64 188.26 PLN $\\

$ W(12)4 = 64 489.04 PLN $\\

$ W(360)4 = 64 606.80 PLN $\\

\subsection{Przykład 8}
Wyznaczymy wartość kapitału 40 000 PLN po 5 latach i 9 miesiącach przy założeniu że roczna stopa procentowa wynosi 6\%, a kapitalizacji odsetek jest a) kwartalna, b) miesięczna.\\

Korzystając(16)\\

\subsubsection{a)}
$ W^4_{(5, 3)} = $
\subsubsection{b)}
$ W^{12}_{(5, 9)} = $

\subsection{Przykład 9}
Przy założeniu miesięcznej kapitalizacji odsetek i rocznych stopach procentowych równych 6\% w pierwszym i drugim roku. 9\% w trzecim i 12\% w czwartym roku wyznaczyć wartość kapitału 100 000 PLN po a)3 latach i 7 miesiącach b) 4 latach.\\

X = kapitał po 3 latach i 7 m\\ 

Y = po 4 latach\\

Wzór (16)\\

$ X = 100 000 * (1+\frac{0,06}{12})^{24} * (1+\frac{0,09}{12})^{12} * (1+\frac{0,12}{12})^7 = 132 183 PLN $\\

$ Y = 100 000 * (1+\frac{0,06}{12})^{24} * (1+\frac{0,09}{12})^{12} * (1+\frac{0,12}{12})^{12} = 138 925,70  PLN $\\

\subsection{Przykład 10}
Przy miesięcznej kapitalizacji odsetek i nominalnej stopie procentowej równej 3\% po 1 roku i 7 miesiącach uzyskano z lokaty 100 PLN odsetek. Jaka była kwota lokaty?\\

Odsetki uzyskane z inwestycji stanowią różnice między wartością kapitału po 1 roku i 7 miesiącach a jego wartością początkową. W = ?\\

$ W^{12}_{(1,7)} - W = 100  \Rightarrow  W = 2 058,29 PLN $\\

%
% Sekcja 4
%

\section{Równoważność stóp pod okresowych przy kapitalizacji złożonej}

Załóżmy że $ r_1 $ i $ r_2 $ są pod okresowymi stopami procentowymi, zaś $ m_1 $ i $ m_2 $ są odpowiadającymi im częstotliwościami kapitalizacji. Stopy $ r_1 $ i $ r_2 $ \textbf{nazywamy równoważnymi w czasie $ l $ lat}, gdzie $ l \in N $, jeżeli przy każdej z nich odsetki od ustalonego kapitału po $ l $ latach są równe.\\


Zauważmy, że równość odsetek po $ l $ latach oznacza równość wartości kapitału po tym czasie. Zatem, uwzględniając wzór (15) otrzymujemy, że podokresowe stopy procentowe $ r_1 $ i $ r_2 $ są równoważne w czasie $ l $ lat, wtedy i tylko wtedy, gdy:

\begin{center}
	\begin{equation}
		(1 + r_1)^{m_1} = (1 + r_2)^{m_2}
	\end{equation}
\end{center}\\

Korzystając ze wzory (17) warunek (19) można przedstawić w następującej równoważnej postaci:

\begin{center}
	\begin{equation}
		(1 + \frac{r_1}{m_1})^{m_1} = (1 + \frac{r_2}{m_2})^{m_2}
	\end{equation}
\end{center}\\

gdzie $ r_1 $ i $ r_2 $ są nominalnymi stopami procentowymi, odpowiednio $ r_1 $ i $ r_2 $.\\

\subsection{Przykład 11}

Wyznaczymy miesięczną stopę procentową równoważną kwartalnej stopie procentowej $ r^{(1)}_{okr} = 4\% $.\\

Ponieważ $ r_1 = \%, m_1 = 4 i m_2 = 12 $ na podstawie (1) mamy:\\

$ (1 + 0,04)^4 = (1 + r_2)^{12} $.\\

Stąd $ r_2 = (1 + 0,04)^{\frac{4}{12}}-1 = 1,3159 \% $

\subsection{Przykład 12}

Wyznaczymy nominalną stopę procentową, która przy kapitalizacji kwartalnej jest równoważna nominalnej stopie $ r_1 = 5 \% $ przy kapitalizacji półrocznej.\\

Korzystając ze wzoru (20) z $ r_1 = 5 \%, m_1 = 2 i m_2 = 4 $ dostajemy:\\

$ (1 + \frac{0,05}{2})^2 = (1 + \frac{r_2}{4})^4 \Rightarrow r_2 =  4,9691 \% $

\newpage

%
% Sekcja 5
%

\section{Efektywna stopa procentowa}

\textbf{Efektywną stopą procentową} nazywa się roczną stopę procentową równoważną danej podokresowej stopie procentowej. Wobec tego, jeśli $ r_{okr} $ jest podokresową stopą procentową, zaś $ m $ jest częstotliwością kapitalizacji, to korzystając z (19) mamy:

\begin{center}
	\begin{equation}
		r^{(m)}_{ef} = (1 + r_{okr})^m - 1
	\end{equation}
\end{center}\\

Z kolei na podstawie (2), efektywną stopę procentową odpowiadającą nominalnej stopie procentowej $ r $ przy m-krotnej kapitalizacji w ciągu roku, wyznacza się z równania:

\begin{center}
	\begin{equation}
		r^{(m)}_{ef} = (1 + \frac{r}{m})^m - 1
	\end{equation}
\end{center}\\

Efektywna stopa procentowa pozwala na zmianę okresu stopy procentowej bez zmiany efektywności kapitalizacji.\\

\subsection{Przykład 13}

Wyznaczymy efektywną stopę procentową odpowiadającą nominalnej stopie procentowej równiej $ 6 \% $ przy kapitalizacji: półrocznej, kwartalnej, miesięcznej, dziennej.\\

Korzystając ze wzoru (22), otrzymujemy\\

$ r^{(m)}_{ef} = (1 + \frac{0,06}{2})^2 - 1 = (1,03)^2 - 1 = 6,09 \% $\\

$ r^{(m)}_{ef} = (1 + \frac{0,06}{4})^4 - 1 = (1,015)^4 - 1 = 6,14 \% $\\

$ r^{(m)}_{ef} = (1 + \frac{0,06}{12})^12 - 1 = (1,005)^12 - 1 = 6,17 \% $\\

$ r^{(m)}_{ef} = (1 + \frac{0,06}{360})^360 - 1 = (1,00016)^360 - 1 = 6,18 \% $\\

Do wyznaczania efektywnej stopy procentowej stopy procentowej można zastosować formułę \textbf{EFEKTYWNA} wbudowaną w pakiecie MS Excel. Jej argumentami są stopa nominalna i liczba okresów.

\subsubsection{a)}

$ EFEKTYWNA(6 \%, 2) = 6.0900 \% $\\

\subsection{Przykład 14}

Wyznaczymy nominalną stopę procentową, której przy: a) kwartalnej, b) miesięcznej kapitalizacji odsetek odpowiada efektywna stopa procentowa równa $ 5 \% $.\\

Wyznaczając $ r $ ze wzoru (22), dostajemy:\\

$ r = m(\sqrt{1 + r^{(m)}_{ef}} - 1) $

\subsubsection{a)}

$ r = 4,9089 \% $

\subsubsection{b)}

$ r = 4,8889 \% $\\

Do wyznaczania nominalnej stopy procentowej można zastosować formułę \textbf{NOMINALNA} z Excela. Jej argumentami są stopa efektywna i liczba okresów.\\

\newpage

%
% Sekcja 6
%

\section{Kapitalizacja ciągła}

Jeżeli przy m-krotnej kapitalizacji w ciągu roku powiększa się liczba okresów, to w granicy przy $ m \rightarrow \infty $ mamy do czynienia z ciągłą kapitalizacją odsetek. W takim przypadku na podstawie wzoru (17) wartość kapitału $ W $ po $ l $ latach można wyznaczyć w następujący sposób.\\

\begin{center}
	\begin{equation}
		W^{(\infty)}_l = We^{l * r}
	\end{equation}
\end{center}\\

Można pokazać, że wzór (23) jest prawdziwy dla $ l> 0 $

\subsection{Przykład 15}

Przy założeniu ciągłej kapitalizacji odsetek i rocznej stopie procentowej $ r = 5 \% $ wyznaczymy wartość kwoty 10 000 PLN po: a) 8 latach, b) 4 latach i 7 miesiącach.

\subsubsection{a)}

ze wzoru (23), $ W = 10 000, l = 8, r = 5 \% $\\

$ W = 10 000 * e^{8*0,05} = 10 000 * e^{0,04} = 14 918,25 $

\subsubsection{b)}

ze wzoru (23), $ W = 10 000, l = 4\frac{7}{12}, r = 5 \% $\\

$ W = 10 000 * e^{4\frac{7}{12}*0,05} = 10 000 * e^{0,2292} = 12 575,94 $

\newpage

%
% Sekcja 7
%

\section{Natężenie procentowe}

W przypadku ciągłej kapitalizacji odsetek efektywną stopę procentową wyznacza się z równania:

\begin{center}
	\begin{equation}
		l + r_{ef} = e^r
	\end{equation}
\end{center}\\

gdzie $ r $ jest stopą nominalną. Zatem:

\begin{center}
	\begin{equation}
		r_{ef} = e^r - 1
	\end{equation}
\end{center}\\

Jeżeli natomiast dana jest efektywna stopa procentowa $ r_{ef} $ to z (24) otrzymujemy stopę nominalną:

\begin{center}
	\begin{equation}
		r = ln(1 + r_{ef})
	\end{equation}
\end{center}\\

Nazywa się natężeniem oprocentowania związanym z efektywną stopą procentową $ r_{ef} $.

\subsection{Przykład 16}

Wyznaczymy natężenie oprocentowania związane z efektywną stopą procentową równą $ 6 \% $.\\

Stosując (26):\\

$ r = ln(1 + 0,06) = Ln(1,06) = 5,83 \% $\\

\newpage

%
% Sekcja 8
%

\section{Dyskonto proste i składane}

Teraz zajmiemy się zagadnieniem ustalania początkowej wartości kapitału na podstawie jego wartości na końcu pewnego okresu. Proces ten nazywa się \textbf{dyskontowaniem}.\\

Dyskonto proste, które jest bezpośrednio związane z prostą kapitalizacją odsetek. W przypadku kapitalizacji prostej na podstawie (2), wartość kapitału początkowego $ W $ po $ n $ latach.\\

W przypadku dyskonta prostego, obecną wartość kapitału $ W $, którą mamy otrzymać (bądź zapłacić) za $ n $ lat wyznacz się na podstawie równości:

\begin{center}
	\begin{equation}
		PV(W) = \frac{W}{1 + nr}
	\end{equation}
\end{center}\\

\textbf{Dyskontem} nazywa się różnicę między wartością kapitału na końcu pewnego ustalonego okresu, a jego wartością na początku tego okresu. Oznaczając dyskonta przed $ D $ i uwzględniając (27) otrzymujemy:

\begin{center}
	\begin{equation}
		D = \frac{nrW}{1 + nr}
	\end{equation}
\end{center}\\

\subsection{Przykład 17}

Zakładając dyskonto proste i przyjmując stopę procentową $ r = 4 \% $ wyznaczyć wartość oraz dyskonto kwoty 50 000 PLN którą mamy otrzymać za 8 lat.\\

Korzystając z (27, 28) $ W = 50 000, r = 0,04, n = 8 $\\

$ PV = \frac{50 000}{1 + 8 * 0,04} = 37 878,79 $ \\

$ D = 50 000 - 37 878,79 = 12 121,21 $\\

W przypadku \textbf{dyskonta składanego}, przy rocznej kapitalizacji odsetek wartość kapitału początkowego $ W $ po $ n $ latach, wyznaczona na podstawie wzoru (10).\\

Zatem obecną wartość kapitału $ W $ która mamy otrzymać (bądź zapłacić) za $ n $ lat wyznacza się z równości:

\begin{center}
	\begin{equation}
		PV(W) = \frac{W}{(1 + r)^n}
	\end{equation}
\end{center}\\

Wielkość $ \frac{1}{(1 + r)} $ nazywa się \textbf{rocznym czynnikiem dyskontującym}. Dyskonto wyraża się w tym przypadku wzorem:

\begin{center}
	\begin{equation}
		PD = W(1 - \frac{1}{(1 + r)^n})
	\end{equation}
\end{center}\\

\subsection{Przykład 18}

Zakładając dyskonto składane i przyjmując stopę procentową $ r = 4 \% $ wyznaczyć wartość oraz dyskonto kwoty 50 000 PLN którą mamy otrzymać za 8 lat.\\

Korzystając z (29, 30) $ W = 50 000, r = 0,04, n = 8 $\\

$ PV = \frac{50 000}{(1 + 0,04)^8}  = 36 534,51$\\

$ D = 50 000 - 36 534,51 = 13 465,49$

\newpage

%
% Sekcja 9
%

\section{Dyskonto przy wielokrotnej kapitalizacji w ciągu roku}

Załóżmy że kapitalizacja odsetek odbywa się m-krotnie w ciągu roku (w równoległych odstępach czasu). Wówczas korzystając ze wzoru (16) obecną wartość $ PV(W) $ kwoty $ W $, którą mamy otrzymać w przyszłości po $ l $ latach i $ n $ spośród $ m $ podokresów $ l + 1  $ roku $ (0 <= n < m $ wyznaczamy wzór:

 
\begin{center}
	\begin{equation}
		PV(W) - \frac{W}{(1 + \frac{r}{m})^{lm + n}}
	\end{equation}
\end{center}\\

Wzór na dyskonto a postać:

\begin{center}
	\begin{equation}
		D = W(1 - \frac{1}{(1 + \frac{r}{m})^{lm + n}})
	\end{equation}
\end{center}\\

W szczególnym przypadku gdy $ n = 0 $ możemy wyznaczyć obecną wartość kwoty $ W $, którą mamy otrzymać po $ l $ latach:

\begin{center}
	\begin{equation}
		PV(W) - \frac{W}{(1 + \frac{r}{m})^{lm}}
	\end{equation}
\end{center}\\

Wzór na dyskonto ma w tym przypadku postać:

\begin{center}
	\begin{equation}
		D = W(1 - \frac{1}{(1 + \frac{r}{m})^{lm}})
	\end{equation}
\end{center}\\

\subsection{Przykład 19}

Przyjmując nominalną stopę procentową $ r = 6\% $ i zakładając kapitalizację a) kwartalną, b) miesięczną, wyznaczyć obecną wartość i dyskonto kwoty 50 000 PLN, którą mamy otrzymać za 2 lata i 3 miesiące.

\subsubsection{a)}

ze wzoru (31)\\

$ PV = \frac{50 000}{(1 + \frac{0,06}{4})^9} = 43 729,61 PLN $\\

$ D = 50 000 - 43 729,61 = 6 270,39 PLN $

\subsubsection{b)}

ze wzoru (31)\\

$ PV = \frac{50 000}{(1 + \frac{0,06}{12})^27} = 43 700, 49 PLN $\\

$ D = 50 000 - 43 700, 49 = 6 299, 51 PLN $

\subsection{Przykład 20}

Przyjmując nominalną stopę procentową równą $ r = 6\% $ i zakładając  kapitalizację a) półroczną, b) miesięczną, c) dzienną, wzyanczyć obecną wartość kwoty 100 000 PLN, którą mamy otrzymać za 3 lata. W każdym przypadku wyznaczyć wartość dyskonta

\subsubsection{a)}

używamy wzoru (33)\\

$ PV = \frac{100 000}{(1 + \frac{0,06}{2})^6} = 83 748,43 PLN $\\

$ D = 100 000 - 83 748,43 = 16 251,57 PLN $

\subsubsection{b)}

$ PV = \frac{100 000}{(1 + \frac{0,06}{12})^{36}} = 83 564,49 PLN $\\

$ D = 100 000 - 83 564,49 = 16 435,51 PLN $

\subsubsection{c)}

$ PV = \frac{100 000}{(1 + \frac{0,06}{360})^{1080}} = 83 528,27 PLN $\\

$ D = 100 000 - 83 528,27 = 16 471,73 PLN $

\subsection{Przykład 21}

Przy założeniu miesięcznej kapitalizacji odsetek obecna wartość kwota 40 000 PLN, którą mamy otrzymać za 2 lata wynosi 36 500 PLN. Wyznaczyć wysokość nominalnej stopy procentowej.\\

Przez $ r $ oznaczmy szukaną nominalną stopę procentową.\\

Korzystając z (33) otrzymujemy równanie na r\\

$ 36 500 = \frac{40 000}{(1 + \frac{r}{12})^{24}} $\\

$  (1 + \frac{r}{12})^{24} = \frac{40 000}{36 500} $\\

$ r = 12(1,0959^{\frac{1}{24}} - 1) = 4,59\% $

\newpage

%
% Sekcja 10
%

\section{Dyskonto przy kapitalizacji ciągłej}

W przypadku ciągłej kapitalizacji odsetek, obecną wartość kwoty $ W $, która mamy otrzymać za $ n $ lat, wyznacza się z równania $ W = PV(W) * e^{rn} $, gdzie $ r $ jest roczną stopą procentową. Stąd:

\begin{center}
	\begin{equation}
		PV(W) = W * e^{-r*n}
	\end{equation}
\end{center}\\

Wzór na dyskonto ma postać

\begin{center}
	\begin{equation}
		D = W(1 = e^{-r * n})
	\end{equation}
\end{center}\\

Wzory (35) i (36) pozostają prawdziwe dla dowolnego $ n > 0 $.

\subsection{Przykład 22}

Zakładając ciągłą kapitalizację odsetek i otrzymując roczną stopę procentową równą $ r = 5\%  $, wyznaczyć obecną wartość i dyskonto kwoty 50 000 PLN, którą mamy otrzymać za 3 lata i 5 miesięcy.\\

Wzór (35)\\

$ PV(W) =  50 000 \cdot e^{-3\frac{5}{12} \cdot 0,05} = 42 148,10 PLN$\\

$ D = 50 000 - 42 148,10 = 7 851,90 PLN $

\subsection{Przykład 23}

Przy założeniu ciągłej kapitalizacji odsetek, obecna wartość kwoty 100 000 PLN, którą mamy otrzymać za 8 lat wynosi 80 000 PLN. Wyznaczyć efektywną stopę procentową.\\

Przez $ r $ oznaczamy szukaną roczną stopę procentową\\

Ze wzoru (35)\\

$ 80 000 = 100 000 \cdot e^{-8r} $\\

$ \frac{80 000}{100 000} = e^{-8r} $\\

$ ln e^{-8r} = ln0,8 $\\

$ -8r = ln 0,8 $\\

$ r = -\frac{1}{8} ln 0,8 $\\

$ r = 0,0279 $\\

$ r = 2,79\% $\\

Efektywna stopa wynosi ze wzoru (25):\\

$ r_{ef} = e^r - 1 = e ^ {0,0279} - 1 = 2,93\% $

\newpage

%
% Sekcja 11
%

\section{Dyskonto handlowe}

Nasze dotychczasowe rozważania dotyczyły dyskonta rzeczywistego, tzn. dyskonta opartego na stopie procentowej. Teraz omówimy dyskonto handlowe. Ograniczymy się przy tym jedynie do dyskonta handlowego prostego, gdyż dyskonto handlowe składane na ogół nie jest wykorzystywane w praktyce. \\

\textbf{Dyskontem handlowym} nazywa się opłatę za pożyczkę obliczoną na podstawie kwoty, którą dłużnik zwóci po ustalonym czasie, zapłaconą w chwili otrzymania pożyczki.\\

Dyskonto handlowe jest również nazywane odsetkami płatnymi z góry, co trafnie oddaje istotę dyskonta, które należy zapłacić w momencie otrzymania pożyczki, a nie przy jej zwrocie.\\

Zasada dyskonta prostego mówi, że dyskonto jest obliczane od kwoty, którą dłużnik zwróci po ustalonym czasie, jest proporcjonalne do tego czasu i jest odejmowane od tej kwoty w momencie udzielania pożyczki.\\

Jeżeli przez $ D $ oznaczymy dyskonto, przez $ P $ początkową wartość pożyczki (tzn. wartość, którą fizycznie dostajemy), a przez $ F $ nominalna wartość pożyczki (to co mamy oddać, na kartce), to otrzymujemy równość:

\begin{center}
	\begin{equation}
		D = F - P
	\end{equation}
\end{center}\\

W dalszym ciągu będziemy zakładać $ F > P > 0 $ 

\newpage

%
% Sekcja 12
%

\section{Stopa dyskontowa}
W przypadku dyskonta handlowego prostego \textbf{stopą dykonstową} nazywa się liczbę okresloną:

\begin{center}
	\begin{equation}
		d = \frac{D - P}{nF}
	\end{equation}
\end{center}\\

gdzie $ n $ oznacza liczbę lat, po której ma nastąpić zwrot pożyczki.

\subsection{Przykład 24}
Wyznaczyć stopę dyskontową pożyczki w kwocie 50 000 PLN udzielonej na 5 lat, jeżeli jej wartość nominalna wynosi 70 000 PLN.\\

Ze wzoru (38)\\

$ d = \frac{70 000 - 50 000}{5 \cdot 70 000} = 5,71\% $

\subsection{Przykład 25}
Obliczyć nominalną wartość 4-letniej pożyczki udzielonej w kwocie 100 000 PLN przy stopie dyskontowej równej $ 5\% $\\

Ze wzoru (38)\\

$ d = \frac{D - P}{nF} $\\

$ dnF = F - P $\\

$ P = F - dnF $\\

$ P = F(1 - dn) $\\

$ F = \frac{P}{1 - dn} $\\

$ F = \frac{100 000}{1 - 4 \cdot 0,05} = 125 000 PLN $

\subsection{Przykład 26}
Przy stopie dyskontowej równej $ r\% $ wyznaczyć początkową wartość dziesięcioletniej pożyczki o nominalnej wartości 200 000 PLN.\\

Ze wzoru (38)\\

$ d = \frac{D - P}{nF} $\\

$ dnF = F - P $\\

$ P = F - dnF $\\

$ P = F(1 - dnF) $\\

$ P = 200 000 - 0,05 \cdot 10 \cdot 200 000 = 120 000 PLN $

\subsection{Przykład 27}
Pożyczka w wysokości 180 000 PLN udzielona na okres 5 lat ma nominalną wartość 240 000 PLN, Obliczyć stopę dyskontową i zbadać jaki wpływ na nominalną wartość pożyczki miałoby podniesienie stopy dyskontowej o 1 pkt procentowy.\\

Ze wzory (38)\\

$ d = \frac{240 000 - 180 000}{5 \cdot 240 000} = 5\% $\\

$ F = \frac{P}{1 - dn} $\\

$ F = \frac{180 000}{1 - 0,06 \cdot 5} = 257 142,90 PLN $\\

Wzrost wartości stopy dyskontowej o 1 pkt procentowy spowodowałby wzrost nominalnej wartości pożyczki z 240 000 do 257 142,90.\\

\newpage

%
% Sekcja 13
%

\section{Zasada równoważności stopy procentowej i stopy dyskontowej}
Zarówno odsetki jak i dyskonto stanowią opłatę za udzieloną pożyczkę. Czyli za możliwość dysponowania określonym kapitałem przez ustalony czas. Ponieważ wielkość te wyznacza się z różnych modeli naturalne wydaje się pytanie, jaki związek między nimi gwarantuje równość opłat za pożyczkę.\\

Roczna stopa procentowa $ r $ i stopa dyskontowa $ d $ nazywają się \textbf{równoważnymi w czasie} $ n $ jeżeli dla dowolnej pożyczki odsetki i dyskonto handlowe wyznaczone przy tych stopach są równe. Tak sformułowana zasada nosi nazwę \textbf{zasady równoważności stopy procentowej i stopy dyskontowej}.








%      Wstawienie nowego wzory z liczbą porządkową
%\begin{center}
%	\begin{equation}
%		
%	\end{equation}
%\end{center}\\



\end{document}