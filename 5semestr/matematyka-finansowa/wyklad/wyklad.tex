\documentclass{article}

\begin{document}

\tableofcontents

\newpage

%
% Sekcja 1
%

\section{Wstęp}

\textbf{Odestkami} - nazywa się kwotę, którą należy zapłacić za prawo użytkowania określonego kapitału. Odsetki są zatem ceną płaconą za wypożyczenie kapitału. Ustala się je w odniesieniu do pewnego ustalonego okresu. Stosunek odsetek do kapitału, który je wygenerował w ustalonym okresie, nazywa się \textbf{okresową stopą procentową.}\\

W praktyce najczęściej mamy do czynienia ze stopami procentowymi ustalonymi dla okresy rocznego. Mówimy wtedy o \textbf{rocznej stopie procentowej}.\\

Jeżeli np. odsetki za 1 rok od pożyczonego kapitału 60 000 PLN wynoszą 1 500 PLN, to roczna stopa procentowa jest równa $ r = \frac{1 500}{60 000} = 2,5 \% $.\\

Powiększenie kapitału o odsetki, które zostały przez niego wygenerowane, nazywa się \textbf{kapitalizacją odsetek}. Czas, w którym odsetki są generowane, nazywa się okresem kapitalizacji. W dalszym ciągu rozważań ograniczymy się do przypadku, gdy odsetki są dopisywane na końcu okresów kapitalizacji. Mówimy wtedy o kapitalizacji z dołu.\\

Wyróżniamy dwa podstawowe rodzaje kapitalizacji: \textbf{prostą i złożoną}.\\

%
% Sekcja 2
%

\section{Kapitalizacja prosta}

W przypadku kapitalizacji prostej odsetki od kapitału oblicza się od kapitału początkowego proporcjonalnie od długości kresu oprocentowania. Oznaczamy przez $ W $ początkową wartość kapitału, przez $ r $ roczną stopę procentową, przez $ I_{n} $ należne za czas $ n $, zaś przez $ W_{n} $ oznaczamy końcową wartość kapitału w czasie $ n $ (w latach).\\

\textbf{Reguła bankowa} – każdy rok ma 360 dni, zaś każdy miesiąc ma 30 dni.

\begin{center}
	\begin{equation}
		I_n = Wnr
	\end{equation}
\end{center}\\

Natomiast wartość końcowa kapitału:

\begin{center}
	\begin{equation}
		W_n = W(1+nr)
	\end{equation}
\end{center}\\

\newpage

\subsection{Przykład 1}
Przy kapitalizacji prostej i rocznej stopie procentowej $ r = 4\% $ wyznaczyć odsetki i końcową wartość kapitału 25 000 PLN po upływie a) 3lat, b) 142dni.\\

\subsubsection{a)}

$ I_n = 25 000 * 3 * 0,04 = 3 000 PLN $

\subsubsection{b)}

$ W_n = 25 000(1 + \frac{142}{360} + 0.04) = 25 394,44 PLN $\\

Załóżmy że czas trwania inwestycji wynosi $ n $ lat i składa się z $ m $ następujących po sobie okresów o długości $ n_1, ...., n_m $. Przyjmijmy że w każdym z nich obowiązuje roczna stopa procentowa, odpowiednio, $ r_1, ..., r_m $. Wtedy wartość kapitału początkowego $ W $ po pierwszym okresie wyniesie: 

\begin{center}
	\begin{equation}
		W_n = W(1 + \Sigma^m_{i=0}r_in_i)
	\end{equation}
\end{center}

\begin{center}
	\begin{equation}
		I_n = W\Sigma^m_{i=0}r_in_i
	\end{equation}
\end{center}\\

\textbf{Przeciętną roczną stopą procentową oprocentowania kapitału} $ W $ w czasie $ n $ nazywa się roczną stopę, przy której kapitał $ W $ generuje w czasie $ n $ odsetki o takiej samej wartości jak przy stopach zmiennych. Definicja ta dotyczy zarówno kapitalizacji prostej i złożonej.\\

Oznaczając przez r(z kreską na górze) przeciętną roczną stopą oprocen, na podstawie wzorów (1) i (4) mamy

\begin{center}
	\begin{equation}
		r = \frac{1}{n}\Sigma^m_{i=1}r_in_i
	\end{equation}
\end{center}\\

Gdyby wszystkie okresy miały jednakową długość to wzór:

\begin{center}
	\begin{equation}
		r = \frac{1}{m}\Sigma^m_{i=1}r_i
	\end{equation}
\end{center}\\

\newpage

\subsection{Przykład 2}
Przez początkowe 4 miesiąca trwania obowiązywała roczna stopa procentowa 6%, przez następnych 5 miesięcy 7%, przez ostatnie 3 miesiące 7,5%.\\

Dane:\\

$ N_1 = \frac{4}{12} $\\

$ N_2 = \frac{5}{12} $\\

$ N_3 = \frac{3}{12} $\\

$ R_1 = 0,06 $\\

$ R_2 = 0,07 $\\

$ R_3 = 0,075 $\\

$ W = 20 000 PLN $

\subsubsection{a)}
Korzystając ze wzoru (3) mamy\\
$ W_3 = 20000(1 + 0.06*\frac{4}{12} + 0.07*\frac{5}{12} + 0.075*\frac{3}{12}) = 21 358,40 PLN $

\subsubsection{b)}
Obliczyć wysokość przeciętnej rocznej stopy oprocentowania\\
Korzystając ze wzory (5) mamy\\
$ r =  0.06*\frac{4}{12} + 0.07*\frac{5}{12} + 0.075*\frac{3}{12} = 6,79\% $\\

Często zdarza się, że stopa procentowa, przy której należy obliczyć odsetki nie jest stopa roczną lecz np. miesięczną lub kwartalną. Okres, po którym odsetki podlegają kapitalizacji nazywa się \textbf{podokresem kapitalizacji}. Stopa procentowa ustalona dla podokresu kapitalizacji nazywa się \textbf{stopą pod okresową. Częstotliwość kapitalizacji} oznacza ile razy odsetki są kapitalizowane w ciągu roku.\\

W dalszym ciągu zakładamy że częstotliwość kapitalizacji wynosi $ m $. Wobec tego każdy rok jest podzielony na $ m $ równych podokresów kapitalizacji.\\

$ m=1 $ – kapitalizacja roczna\\

$ m=2 $ – kapitalizacja półroczna\\

$ m=4 $ – kapitalizacja kwartalna\\

$ m=12 $ – kapitalizacja miesięczna\\

$ m=360 $ – kapitalizacja dobowa(dzienna)\\

Jeżeli $ r_{okr} $ jest stopą podokresową, to zgodnie z zasadą oprocentowania prostego odsetki od kapitału $ W $ po upływie $ k $ podokresów wyznacza się ze wzoru

\begin{center}
	\begin{equation}
		I_k = Wkr_{okr}
	\end{equation}
\end{center}\\

Natomiast końcowa wartość kapitału W po upływie $ k $:\\

\begin{center}
	\begin{equation}
		W_k = W(1 + kr_{okr})
	\end{equation}
\end{center}\\

Załóżmy że $ r_1 $ i $ r_2 $ są podokresowymi stopami procentowymi, zaś $ m_1 $ i $ m_2 $ są odpowiadającymi im częstotliwościami kapitalizacji. Stopy $ r_1 $ i $ r_2 $ nazywamy równoważnymi w czasie $ n $, jeżeli przy każdej z nich odsetki od ustalonego kapitału po czasie $ n $ są równe.\\

Korzystając z (7) mamy:\\

\begin{center}
	\begin{equation}
		m_1 * r_1 = m_2 * r_2
	\end{equation}
\end{center}\\

Z (9) stopy pod okresowe są wtedy i tylko wtedy ich stosunek jest równy stosunkowi długości odpowiadających im po okresów. Takie stopy pod okresowe nazywają się \textbf{proporcjonalnymi}.

\subsection{Przykład 3}
Kwartalna stopa oprocentowania prostego wynosi 6%. Wyznaczyć następujące równoważne stopy oprocentowania prostego: 

\subsubsection{a) roczna}
$ 6 * 4 = 24\% $
\subsubsection{b) miesięczna}
$ 6 / 3 = 2\% $
\subsubsection{c) tygodniowa}
$ 6 / 12 = 0,5\% $

\newpage

%
% Sekcja 3
%

\section{Kapitalizacja złożona}
W przypadku \textbf{kapitalizacji złożonej} odsetki oblicza się za każdy okres równy okresowi kapitalizacji i kapitalizuje się je na koniec tego okresu. Załóżmy, że kwota $ W $ została ulokowana na rachunku z roczną stopą procentową równą $ r $. W przypadku kapitalizacji złożonej dochód przynosi początkowy kapitał wraz z odsetkami uzyskanymi na koniec poprzedniego okresu kapitalizacji. Przez $ I_n $ oznaczmy odsetki należne po czasie $ n $, zaś przez $ W_n $ oznaczmy wartość kapitału po $ n $ latach. Wtedy: $ W_1 = w(1+r) $\\

\begin{center}
	\begin{equation}
		W_n = W(1+r)^n
	\end{equation}
\end{center}\\

Liczba $ (1+r)^n $ nazywa się \textbf{czynnikiem wartości przyszłej} w kapitalizacji złożonej.\\

Odsetki po okresie n lat wynoszą:\\

\begin{center}
	\begin{equation}
		I_n = W((1+r)^n - 1)
	\end{equation}
\end{center}\\

\subsection{Przykład 4}
Przy założeniu kapitalizacji złożonej i rocznej stopie procentowej r = 5\%, wyznaczymy wartość kapitału 40 000 PLN i odsetki po upływie 4 lat.\\

$ W_n = 40 000(1+ 0,05)^4 = 48 620 PLN $\\

$ I_n = 48 620 - 40 000 - 8 620 PLN $\\

$ I_n = 40 000((1 + 0,05)^4 - 1) = 8 620 PLN $\\

Podobnie jak w przypadku kapitalizacji prostej w kapitalizacji złożonej, możemy dopuścić zmienne stopy procentowe w kolejnych latach trwania inwestycji. Przyjmijmy, że w kolejnych latach stopy procentowe są równe $ r_1, r_2, ..., 4_n $ gdzie $ n $ jest liczą lat trwania inwestycji. Wtedy wartość początkowego kapitału $ W $ po pierwszym roku wyniesie. $ W_1 = W(1+r_1) $, po drugim $ W_2 = W(1+r_1)(1+r_2) $\\

Wartość kapitału po $ n $ latach: 

\begin{center}
	\begin{equation}
		W_n = W\Pi^n_{i=1}(1 + r_i)
	\end{equation}
\end{center}\\

\begin{center}
	\begin{equation}
		I_n = W(\Pi^n_{i=1}(1 + r_i) - 1)
	\end{equation}
\end{center}\\

Przeciętna roczna stopa oprocentowania w przypadku kapitalizacji złożonej:\\

\begin{center}
	\begin{equation}
		r = (\Pi^n_{i=1}(1 + r_1))^{\frac{1}{n}} - 1
	\end{equation}
\end{center}\\

\subsection{Przykład 5}
Kapitał 20 000 PLN został ulokowany na okres 5 lat. Przy założeniu kapitalizacji złożonej i rocznej stopie procentowej równej w kolejnych latach, 5\%, 6\%, 5\%, 4\%, 7\%, wyznaczymy wartości kapitału na koniec kolejnych lat oraz przeciętną roczną stopę oprocentowania tego kapitału w czasie 5 lat. \\

$ W_1 = 21 000 PLN $\\

$ W_5 = 20 000(1+0.05)(1+0.06)(1+0.05)(1+0.04)(1+0.07) = 26 009.47 PLN $ \\

$ r = ((1+0.05)(1+0.06)(1+0.05)(1+0.04)(1+0.07))^{\frac{1}{5}} – 1 = 5.40\% $\\

Niech $ r_{okr} $ będzie stopą pod okresową. Przy założeniu kapitalizacji złożonej, przyszła wartość kwoty $ W $ po l latach i $ n $ spośród $ m $ pod okresów $ l+1 $ roku, gdzie $ 0 \leq n < m $ wynosi:

\begin{center}
	\begin{equation}
		W^{(m)}_{(l, n)} = W(1 + r_{okr})^{l * m + n}
	\end{equation}
\end{center}\\

\subsection{Przykład 6}
Zakładając kapitalizację a) półroczną, b) kwartalną c) miesięczną i przyjmując stopę pod okresową $ r_{okr} $ = 2\% wyznaczyć przyszłą wartość kapitału 20 000 PLN po 2 latach i 6 miesiącach.

\subsubsection{a)}
$ W^{(2)}_{(2, 1)} = W(1 + 0,02)^{2 * 2 + 1} = 22 081,62 PLN $\\
\subsubsection{b)}
$ W^{(4)}_{(2, 2)} = W(1 + 0,02)^{10} = 24 379,89 PLN $\\
\subsubsection{c)}
$ W^{(12)}_{(2, 6)} = W(1 + 0,02)^{30} = 36 227,23 PLN $\\

Roczna stopa procentowa $ r $ proporcjonalna do danej stopy pod okresowej $ r_{okr} $ nazywa się \textbf{stopą nominalną}. ( wyliczyć roczną stopę, np. jak miesięczna jest 1\% to roczna jest 12\% itp.)

\begin{center}
	\begin{equation}
		W^{(m)}_{(l, n)} = W(1 + \frac{r}{m})^{l*m+n}
	\end{equation}
\end{center}\\

Przyjmując n = 0 wtedy:\\

\begin{center}
	\begin{equation}
		W^{(m)}_{l} = W(1 + \frac{r}{m})^{l*m}
	\end{equation}
\end{center}\\

Liczbę:\\

\begin{center}
	\begin{equation}
		R_m = (1 + \frac{r}{m})^m
	\end{equation}
\end{center}\\

Nazywa się rocznym \textbf{czynnikiem oprocentowania}.

\subsection{Przykład 7}
Kapitał w wysokości 40 000 PN został ulokowany na rachunku z nominalną stopą procentową równą 12\%. Zakładając kapitalizację, roczną, półroczną, kwartalną, miesięczną oraz dzienna, wyznaczyć przyszłą wartość kapitału po 4 latach.\\

Ze wzory (17)\\

$ W(1)4 = 62 940,77 PLN $\\ 

$ W(2)4 = 63 753.92 PLN $\\

$ W(4)4 = 64 188.26 PLN $\\

$ W(12)4 = 64 489.04 PLN $\\

$ W(360)4 = 64 606.80 PLN $\\

\subsection{Przykład 8}
Wyznaczymy wartość kapitału 40 000 PLN po 5 latach i 9 miesiącach przy założeniu że roczna stopa procentowa wynosi 6\%, a kapitalizacji odsetek jest a) kwartalna, b) miesięczna.\\

Korzystając(16)\\

\subsubsection{a)}
$ W^4_{(5, 3)} = $
\subsubsection{b)}
$ W^{12}_{(5, 9)} = $

\subsection{Przykład 9}
Przy założeniu miesięcznej kapitalizacji odsetek i rocznych stopach procentowych równych 6\% w pierwszym i drugim roku. 9\% w trzecim i 12\% w czwartym roku wyznaczyć wartość kapitału 100 000 PLN po a)3 latach i 7 miesiącach b) 4 latach.\\

X = kapitał po 3 latach i 7 m\\ 

Y = po 4 latach\\

Wzór (16)\\

$ X = 100 000 * (1+\frac{0,06}{12})^{24} * (1+\frac{0,09}{12})^{12} * (1+\frac{0,12}{12})^7 = 132 183 PLN $\\

$ Y = 100 000 * (1+\frac{0,06}{12})^{24} * (1+\frac{0,09}{12})^{12} * (1+\frac{0,12}{12})^{12} = 138 925,70  PLN $\\

\subsection{Przykład 10}
Przy miesięcznej kapitalizacji odsetek i nominalnej stopie procentowej równej 3\% po 1 roku i 7 miesiącach uzyskano z lokaty 100 PLN odsetek. Jaka była kwota lokaty?\\

Odsetki uzyskane z inwestycji stanowią różnice między wartością kapitału po 1 roku i 7 miesiącach a jego wartością początkową. W = ?\\

$ W^{12}_{(1,7)} - W = 100  \Rightarrow  W = 2 058,29 PLN $\\

%
% Sekcja 4
%

\section{Równoważność stóp pod okresowych przy kapitalizacji złożonej}

Załóżmy że $ r_1 $ i $ r_2 $ są pod okresowymi stopami procentowymi, zaś $ m_1 $ i $ m_2 $ są odpowiadającymi im częstotliwościami kapitalizacji. Stopy $ r_1 $ i $ r_2 $ \textbf{nazywamy równoważnymi w czasie $ l $ lat}, gdzie $ l \in N $, jeżeli przy każdej z nich odsetki od ustalonego kapitału po $ l $ latach są równe.\\


Zauważmy, że równość odsetek po $ l $ latach oznacza równość wartości kapitału po tym czasie. Zatem, uwzględniając wzór (15) otrzymujemy, że podokresowe stopy procentowe $ r_1 $ i $ r_2 $ są równoważne w czasie $ l $ lat, wtedy i tylko wtedy, gdy:

\begin{center}
	\begin{equation}
		(1 + r_1)^{m_1} = (1 + r_2)^{m_2}
	\end{equation}
\end{center}\\

Korzystając ze wzory (17) warunek (19) można przedstawić w następującej równoważnej postaci:

\begin{center}
	\begin{equation}
		(1 + \frac{r_1}{m_1})^{m_1} = (1 + \frac{r_2}{m_2})^{m_2}
	\end{equation}
\end{center}\\

gdzie $ r_1 $ i $ r_2 $ są nominalnymi stopami procentowymi, odpowiednio $ r_1 $ i $ r_2 $.\\

\subsection{Przykład 11}

Wyznaczymy miesięczną stopę procentową równoważną kwartalnej stopie procentowej $ r^{(1)}_{okr} = 4\% $.\\

Ponieważ $ r_1 = \%, m_1 = 4 i m_2 = 12 $ na podstawie (1) mamy:\\

$ (1 + 0,04)^4 = (1 + r_2)^{12} $.\\

Stąd $ r_2 = (1 + 0,04)^{\frac{4}{12}}-1 = 1,3159 \% $

\subsection{Przykład 12}

Wyznaczymy nominalną stopę procentową, która przy kapitalizacji kwartalnej jest równoważna nominalnej stopie $ r_1 = 5 \% $ przy kapitalizacji półrocznej.\\

Korzystając ze wzoru (20) z $ r_1 = 5 \%, m_1 = 2 i m_2 = 4 $ dostajemy:\\

$ (1 + \frac{0,05}{2})^2 = (1 + \frac{r_2}{4})^4 \Rightarrow r_2 =  4,9691 \% $

\newpage

%
% Sekcja 5
%

\section{Efektywna stopa procentowa}

\textbf{Efektywną stopą procentową} nazywa się roczną stopę procentową równoważną danej podokresowej stopie procentowej. Wobec tego, jeśli $ r_{okr} $ jest podokresową stopą procentową, zaś $ m $ jest częstotliwością kapitalizacji, to korzystając z (19) mamy:

\begin{center}
	\begin{equation}
		r^{(m)}_{ef} = (1 + r_{okr})^m - 1
	\end{equation}
\end{center}\\

Z kolei na podstawie (2), efektywną stopę procentową odpowiadającą nominalnej stopie procentowej $ r $ przy m-krotnej kapitalizacji w ciągu roku, wyznacza się z równania:

\begin{center}
	\begin{equation}
		r^{(m)}_{ef} = (1 + \frac{r}{m})^m - 1
	\end{equation}
\end{center}\\

Efektywna stopa procentowa pozwala na zmianę okresu stopy procentowej bez zmiany efektywności kapitalizacji.\\

\subsection{Przykład 13}

Wyznaczymy efektywną stopę procentową odpowiadającą nominalnej stopie procentowej równiej $ 6 \% $ przy kapitalizacji: półrocznej, kwartalnej, miesięcznej, dziennej.\\

Korzystając ze wzoru (22), otrzymujemy\\

$ r^{(m)}_{ef} = (1 + \frac{0,06}{2})^2 - 1 = (1,03)^2 - 1 = 6,09 \% $\\

$ r^{(m)}_{ef} = (1 + \frac{0,06}{4})^4 - 1 = (1,015)^4 - 1 = 6,14 \% $\\

$ r^{(m)}_{ef} = (1 + \frac{0,06}{12})^12 - 1 = (1,005)^12 - 1 = 6,17 \% $\\

$ r^{(m)}_{ef} = (1 + \frac{0,06}{360})^360 - 1 = (1,00016)^360 - 1 = 6,18 \% $\\

Do wyznaczania efektywnej stopy procentowej stopy procentowej można zastosować formułę \textbf{EFEKTYWNA} wbudowaną w pakiecie MS Excel. Jej argumentami są stopa nominalna i liczba okresów.

\subsubsection{a)}

$ EFEKTYWNA(6 \%, 2) = 6.0900 \% $\\

\subsection{Przykład 14}

Wyznaczymy nominalną stopę procentową, której przy: a) kwartalnej, b) miesięcznej kapitalizacji odsetek odpowiada efektywna stopa procentowa równa $ 5 \% $.\\

Wyznaczając $ r $ ze wzoru (22), dostajemy:\\

$ r = m(\sqrt{1 + r^{(m)}_{ef}} - 1) $

\subsubsection{a)}

$ r = 4,9089 \% $

\subsubsection{b)}

$ r = 4,8889 \% $\\

Do wyznaczania nominalnej stopy procentowej można zastosować formułę \textbf{NOMINALNA} z Excela. Jej argumentami są stopa efektywna i liczba okresów.\\

\newpage

%
% Sekcja 6
%

\section{Kapitalizacja ciągła}

Jeżeli przy m-krotnej kapitalizacji w ciągu roku powiększa się liczba okresów, to w granicy przy $ m \rightarrow \infty $ mamy do czynienia z ciągłą kapitalizacją odsetek. W takim przypadku na podstawie wzoru (17) wartość kapitału $ W $ po $ l $ latach można wyznaczyć w następujący sposób.\\

\begin{center}
	\begin{equation}
		W^{(\infty)}_l = We^{l * r}
	\end{equation}
\end{center}\\

Można pokazać, że wzór (23) jest prawdziwy dla $ l> 0 $

\subsection{Przykład 15}

Przy założeniu ciągłej kapitalizacji odsetek i rocznej stopie procentowej $ r = 5 \% $ wyznaczymy wartość kwoty 10 000 PLN po: a) 8 latach, b) 4 latach i 7 miesiącach.

\subsubsection{a)}

ze wzoru (23), $ W = 10 000, l = 8, r = 5 \% $\\

$ W = 10 000 * e^{8*0,05} = 10 000 * e^{0,04} = 14 918,25 $

\subsubsection{b)}

ze wzoru (23), $ W = 10 000, l = 4\frac{7}{12}, r = 5 \% $\\

$ W = 10 000 * e^{4\frac{7}{12}*0,05} = 10 000 * e^{0,2292} = 12 575,94 $

\newpage

%
% Sekcja 7
%

\section{Natężenie procentowe}

W przypadku ciągłej kapitalizacji odsetek efektywną stopę procentową wyznacza się z równania:

\begin{center}
	\begin{equation}
		l + r_{ef} = e^r
	\end{equation}
\end{center}\\

gdzie $ r $ jest stopą nominalną. Zatem:

\begin{center}
	\begin{equation}
		r_{ef} = e^r - 1
	\end{equation}
\end{center}\\

Jeżeli natomiast dana jest efektywna stopa procentowa $ r_{ef} $ to z (24) otrzymujemy stopę nominalną:

\begin{center}
	\begin{equation}
		r = ln(1 + r_{ef})
	\end{equation}
\end{center}\\

Nazywa się natężeniem oprocentowania związanym z efektywną stopą procentową $ r_{ef} $.

\subsection{Przykład 16}

Wyznaczymy natężenie oprocentowania związane z efektywną stopą procentową równą $ 6 \% $.\\

Stosując (26):\\

$ r = ln(1 + 0,06) = Ln(1,06) = 5,83 \% $\\

\newpage

%
% Sekcja 8
%

\section{Dyskonto proste i składane}

Teraz zajmiemy się zagadnieniem ustalania początkowej wartości kapitału na podstawie jego wartości na końcu pewnego okresu. Proces ten nazywa się \textbf{dyskontowaniem}.\\

Dyskonto proste, które jest bezpośrednio związane z prostą kapitalizacją odsetek. W przypadku kapitalizacji prostej na podstawie (2), wartość kapitału początkowego $ W $ po $ n $ latach.\\

W przypadku dyskonta prostego, obecną wartość kapitału $ W $, którą mamy otrzymać (bądź zapłacić) za $ n $ lat wyznacz się na podstawie równości:

\begin{center}
	\begin{equation}
		PV(W) = \frac{W}{1 + nr}
	\end{equation}
\end{center}\\

\textbf{Dyskontem} nazywa się różnicę między wartością kapitału na końcu pewnego ustalonego okresu, a jego wartością na początku tego okresu. Oznaczając dyskonta przed $ D $ i uwzględniając (27) otrzymujemy:

\begin{center}
	\begin{equation}
		D = \frac{nrW}{1 + nr}
	\end{equation}
\end{center}\\

\subsection{Przykład 17}

Zakładając dyskonto proste i przyjmując stopę procentową $ r = 4 \% $ wyznaczyć wartość oraz dyskonto kwoty 50 000 PLN którą mamy otrzymać za 8 lat.\\

Korzystając z (27, 28) $ W = 50 000, r = 0,04, n = 8 $\\

$ PV = \frac{50 000}{1 + 8 * 0,04} = 37 878,79 $ \\

$ D = 50 000 - 37 878,79 = 12 121,21 $\\

W przypadku \textbf{dyskonta składanego}, przy rocznej kapitalizacji odsetek wartość kapitału początkowego $ W $ po $ n $ latach, wyznaczona na podstawie wzoru (10).\\

Zatem obecną wartość kapitału $ W $ która mamy otrzymać (bądź zapłacić) za $ n $ lat wyznacza się z równości:

\begin{center}
	\begin{equation}
		PV(W) = \frac{W}{(1 + r)^n}
	\end{equation}
\end{center}\\

Wielkość $ \frac{1}{(1 + r)} $ nazywa się \textbf{rocznym czynnikiem dyskontującym}. Dyskonto wyraża się w tym przypadku wzorem:

\begin{center}
	\begin{equation}
		PD = W(1 - \frac{1}{(1 + r)^n})
	\end{equation}
\end{center}\\

\subsection{Przykład 18}

Zakładając dyskonto składane i przyjmując stopę procentową $ r = 4 \% $ wyznaczyć wartość oraz dyskonto kwoty 50 000 PLN którą mamy otrzymać za 8 lat.\\

Korzystając z (29, 30) $ W = 50 000, r = 0,04, n = 8 $\\

$ PV = \frac{50 000}{(1 + 0,04)^8}  = 36 534,51$\\

$ D = 50 000 - 36 534,51 = 13 465,49$

\newpage

%
% Sekcja 9
%

\section{Dyskonto przy wielokrotnej kapitalizacji w ciągu roku}

Załóżmy że kapitalizacja odsetek odbywa się m-krotnie w ciągu roku (w równoległych odstępach czasu). Wówczas korzystając ze wzoru (16) obecną wartość $ PV(W) $ kwoty $ W $, którą mamy otrzymać w przyszłości po $ l $ latach i $ n $ spośród $ m $ podokresów $ l + 1  $ roku $ (0 <= n < m $ wyznaczamy wzór:

 
\begin{center}
	\begin{equation}
		PV(W) - \frac{W}{(1 + \frac{r}{m})^{lm + n}}
	\end{equation}
\end{center}\\

Wzór na dyskonto a postać:

\begin{center}
	\begin{equation}
		D = W(1 - \frac{1}{(1 + \frac{r}{m})^{lm + n}})
	\end{equation}
\end{center}\\

W szczególnym przypadku gdy $ n = 0 $ możemy wyznaczyć obecną wartość kwoty $ W $, którą mamy otrzymać po $ l $ latach:

\begin{center}
	\begin{equation}
		PV(W) - \frac{W}{(1 + \frac{r}{m})^{lm}}
	\end{equation}
\end{center}\\

Wzór na dyskonto ma w tym przypadku postać:

\begin{center}
	\begin{equation}
		D = W(1 - \frac{1}{(1 + \frac{r}{m})^{lm}})
	\end{equation}
\end{center}\\

\subsection{Przykład 19}

Przyjmując nominalną stopę procentową $ r = 6\% $ i zakładając kapitalizację a) kwartalną, b) miesięczną, wyznaczyć obecną wartość i dyskonto kwoty 50 000 PLN, którą mamy otrzymać za 2 lata i 3 miesiące.

\subsubsection{a)}

ze wzoru (31)\\

$ PV = \frac{50 000}{(1 + \frac{0,06}{4})^9} = 43 729,61 PLN $\\

$ D = 50 000 - 43 729,61 = 6 270,39 PLN $

\subsubsection{b)}

ze wzoru (31)\\

$ PV = \frac{50 000}{(1 + \frac{0,06}{12})^27} = 43 700, 49 PLN $\\

$ D = 50 000 - 43 700, 49 = 6 299, 51 PLN $

\subsection{Przykład 20}

Przyjmując nominalną stopę procentową równą $ r = 6\% $ i zakładając  kapitalizację a) półroczną, b) miesięczną, c) dzienną, wzyanczyć obecną wartość kwoty 100 000 PLN, którą mamy otrzymać za 3 lata. W każdym przypadku wyznaczyć wartość dyskonta

\subsubsection{a)}

używamy wzoru (33)\\

$ PV = \frac{100 000}{(1 + \frac{0,06}{2})^6} = 83 748,43 PLN $\\

$ D = 100 000 - 83 748,43 = 16 251,57 PLN $

\subsubsection{b)}

$ PV = \frac{100 000}{(1 + \frac{0,06}{12})^{36}} = 83 564,49 PLN $\\

$ D = 100 000 - 83 564,49 = 16 435,51 PLN $

\subsubsection{c)}

$ PV = \frac{100 000}{(1 + \frac{0,06}{360})^{1080}} = 83 528,27 PLN $\\

$ D = 100 000 - 83 528,27 = 16 471,73 PLN $

\subsection{Przykład 21}

Przy założeniu miesięcznej kapitalizacji odsetek obecna wartość kwota 40 000 PLN, którą mamy otrzymać za 2 lata wynosi 36 500 PLN. Wyznaczyć wysokość nominalnej stopy procentowej.\\

Przez $ r $ oznaczmy szukaną nominalną stopę procentową.\\

Korzystając z (33) otrzymujemy równanie na r\\

$ 36 500 = \frac{40 000}{(1 + \frac{r}{12})^{24}} $\\

$  (1 + \frac{r}{12})^{24} = \frac{40 000}{36 500} $\\

$ r = 12(1,0959^{\frac{1}{24}} - 1) = 4,59\% $

\newpage

%
% Sekcja 10
%

\section{Dyskonto przy kapitalizacji ciągłej}

W przypadku ciągłej kapitalizacji odsetek, obecną wartość kwoty $ W $, która mamy otrzymać za $ n $ lat, wyznacza się z równania $ W = PV(W) * e^{rn} $, gdzie $ r $ jest roczną stopą procentową. Stąd:

\begin{center}
	\begin{equation}
		PV(W) = W * e^{-r*n}
	\end{equation}
\end{center}\\

Wzór na dyskonto ma postać

\begin{center}
	\begin{equation}
		D = W(1 = e^{-r * n})
	\end{equation}
\end{center}\\

Wzory (35) i (36) pozostają prawdziwe dla dowolnego $ n > 0 $.

\subsection{Przykład 22}

Zakładając ciągłą kapitalizację odsetek i otrzymując roczną stopę procentową równą $ r = 5\%  $, wyznaczyć obecną wartość i dyskonto kwoty 50 000 PLN, którą mamy otrzymać za 3 lata i 5 miesięcy.\\

Wzór (35)\\

$ PV(W) =  50 000 \cdot e^{-3\frac{5}{12} \cdot 0,05} = 42 148,10 PLN$\\

$ D = 50 000 - 42 148,10 = 7 851,90 PLN $

\subsection{Przykład 23}

Przy założeniu ciągłej kapitalizacji odsetek, obecna wartość kwoty 100 000 PLN, którą mamy otrzymać za 8 lat wynosi 80 000 PLN. Wyznaczyć efektywną stopę procentową.\\

Przez $ r $ oznaczamy szukaną roczną stopę procentową\\

Ze wzoru (35)\\

$ 80 000 = 100 000 \cdot e^{-8r} $\\

$ \frac{80 000}{100 000} = e^{-8r} $\\

$ ln e^{-8r} = ln0,8 $\\

$ -8r = ln 0,8 $\\

$ r = -\frac{1}{8} ln 0,8 $\\

$ r = 0,0279 $\\

$ r = 2,79\% $\\

Efektywna stopa wynosi ze wzoru (25):\\

$ r_{ef} = e^r - 1 = e ^ {0,0279} - 1 = 2,93\% $

\newpage

%
% Sekcja 11
%

\section{Dyskonto handlowe}

Nasze dotychczasowe rozważania dotyczyły dyskonta rzeczywistego, tzn. dyskonta opartego na stopie procentowej. Teraz omówimy dyskonto handlowe. Ograniczymy się przy tym jedynie do dyskonta handlowego prostego, gdyż dyskonto handlowe składane na ogół nie jest wykorzystywane w praktyce. \\

\textbf{Dyskontem handlowym} nazywa się opłatę za pożyczkę obliczoną na podstawie kwoty, którą dłużnik zwóci po ustalonym czasie, zapłaconą w chwili otrzymania pożyczki.\\

Dyskonto handlowe jest również nazywane odsetkami płatnymi z góry, co trafnie oddaje istotę dyskonta, które należy zapłacić w momencie otrzymania pożyczki, a nie przy jej zwrocie.\\

Zasada dyskonta prostego mówi, że dyskonto jest obliczane od kwoty, którą dłużnik zwróci po ustalonym czasie, jest proporcjonalne do tego czasu i jest odejmowane od tej kwoty w momencie udzielania pożyczki.\\

Jeżeli przez $ D $ oznaczymy dyskonto, przez $ P $ początkową wartość pożyczki (tzn. wartość, którą fizycznie dostajemy), a przez $ F $ nominalna wartość pożyczki (to co mamy oddać, na kartce), to otrzymujemy równość:

\begin{center}
	\begin{equation}
		D = F - P
	\end{equation}
\end{center}\\

W dalszym ciągu będziemy zakładać $ F > P > 0 $ 

\newpage

%
% Sekcja 12
%

\section{Stopa dyskontowa}
W przypadku dyskonta handlowego prostego \textbf{stopą dykonstową} nazywa się liczbę okresloną:

\begin{center}
	\begin{equation}
		d = \frac{D - P}{nF}
	\end{equation}
\end{center}\\

gdzie $ n $ oznacza liczbę lat, po której ma nastąpić zwrot pożyczki.

\subsection{Przykład 24}
Wyznaczyć stopę dyskontową pożyczki w kwocie 50 000 PLN udzielonej na 5 lat, jeżeli jej wartość nominalna wynosi 70 000 PLN.\\

Ze wzoru (38)\\

$ d = \frac{70 000 - 50 000}{5 \cdot 70 000} = 5,71\% $

\subsection{Przykład 25}
Obliczyć nominalną wartość 4-letniej pożyczki udzielonej w kwocie 100 000 PLN przy stopie dyskontowej równej $ 5\% $\\

Ze wzoru (38)\\

$ d = \frac{D - P}{nF} $\\

$ dnF = F - P $\\

$ P = F - dnF $\\

$ P = F(1 - dn) $\\

$ F = \frac{P}{1 - dn} $\\

$ F = \frac{100 000}{1 - 4 \cdot 0,05} = 125 000 PLN $

\subsection{Przykład 26}
Przy stopie dyskontowej równej $ r\% $ wyznaczyć początkową wartość dziesięcioletniej pożyczki o nominalnej wartości 200 000 PLN.\\

Ze wzoru (38)\\

$ d = \frac{D - P}{nF} $\\

$ dnF = F - P $\\

$ P = F - dnF $\\

$ P = F(1 - dnF) $\\

$ P = 200 000 - 0,05 \cdot 10 \cdot 200 000 = 120 000 PLN $

\subsection{Przykład 27}
Pożyczka w wysokości 180 000 PLN udzielona na okres 5 lat ma nominalną wartość 240 000 PLN, Obliczyć stopę dyskontową i zbadać jaki wpływ na nominalną wartość pożyczki miałoby podniesienie stopy dyskontowej o 1 pkt procentowy.\\

Ze wzory (38)\\

$ d = \frac{240 000 - 180 000}{5 \cdot 240 000} = 5\% $\\

$ F = \frac{P}{1 - dn} $\\

$ F = \frac{180 000}{1 - 0,06 \cdot 5} = 257 142,90 PLN $\\

Wzrost wartości stopy dyskontowej o 1 pkt procentowy spowodowałby wzrost nominalnej wartości pożyczki z 240 000 do 257 142,90.\\

\newpage

%
% Sekcja 13
%

\section{Zasada równoważności stopy procentowej i stopy dyskontowej}
Zarówno odsetki jak i dyskonto stanowią opłatę za udzieloną pożyczkę. Czyli za możliwość dysponowania określonym kapitałem przez ustalony czas. Ponieważ wielkość te wyznacza się z różnych modeli naturalne wydaje się pytanie, jaki związek między nimi gwarantuje równość opłat za pożyczkę.\\

Roczna stopa procentowa $ r $ i stopa dyskontowa $ d $ nazywają się \textbf{równoważnymi w czasie} $ n $ jeżeli dla dowolnej pożyczki odsetki i dyskonto handlowe wyznaczone przy tych stopach są równe. Tak sformułowana zasada nosi nazwę \textbf{zasady równoważności stopy procentowej i stopy dyskontowej}.\\

Załóżmy, że wartość pozyczki wynosi $ P $. Przy założeniu kapitalizacji prostej i rocznej stopie procentowej równej $ r $, na podstawie wzoru (2) wartość kwoty $ P $ po $ n $ latach wynosi: $ P_n = P(1 + nr) $. Zatem odsetki sa równe $ P_n - P = nrP $.\\

Z drugiej storny ze wzoru (37) i (38) mamy $ D = F - P = ndF $. Na podstawie zasady równoważności stopy procentowej i stopy dyskonotej otrzymujemy więc równość $ ndF = nrP $. Stąd wynika, że $ dF = rP $. ze wzoru (38) dostajemy $ F = \frac{P}{1 - dn} $. Na końcu orzymujemy:

\begin{center}
	\begin{equation}
		r = \frac{d}{1 - dn}
	\end{equation}
\end{center}\\

\subsection{Przykład 28}
Wyznaczymy stopę procentową równoważną w czasie 6 lat stopie dyskontowej 8\%.\\

Ze wzoru (39) otrzymujemy $ r = \frac{0,08}{1 - 0.08 \cdot 6} = 15,38\% $

\subsection{Przykład 29}
Wyznaczymy stopę dyskontową równoważną w czasie 8 lat stopie procentowej 4\%.\\

Ze wzoru (39) otrzymujemy $ d = \frac{r}{1 + rn} $, zatem $ d = \frac{0,04}{1 + 8 \cdot 0,04} = 3,03\% $

\subsection{Przykład 30}
Wyznaczymy czas, w którym stopa dyskontowa równa 5\% jest równoważna stopie procentowej równiej 8\%.\\

Ze wzoru (39) otrzymujemy $ n = \frac{1}{d} - \frac{1}{r} $, zatem $ n = \frac{1}{0,05} - \frac{1}{0,08} = 7,5 $. Podane stopy są więc równoważne w czasie 7 lat i 6 miesięcy.\\

%
% Sekcja 14
%

\newpage

\section{Weksle}

Dyskonto handlowe znajduje zastosowanie w m.in. w rachunku weklski. \textbf{Weksel} to zobowiązanie do zapłaty określonej kwoty w ustalonym terminie. Ma on formę dokumentu sprecyzowaną odpowiednimi przepisami prawa. Kwota do zapłaty, której zobowiązuje weksel, nazywa się jego \textbf{wartością nominalną}. Termin, w którym weksel ma być spłacony nazywa się \textbf{terminem wykupu} weksla. Kwota nominalna pomniejszona o dyskonto nazywa sie \textbf{wartością aktualną} weksla. 

\subsection{Przykład 31}
Zobowiązanie do zapłaty za dostarczony towar o wartości 390 000 PLN ma formę weksla podpisanego w dniu 5 maja na sumę 400 000 PLN z terminem wykupu 5 sierpnia tego samego roku. Mamy zatem $ P = 390 000, F = 400 000, n = \frac{90}{360} $. Stąd $ D = F - P = 10 000 $(dyskonto), czyli na podstawie (39) stopa dyskontowa wynosi, $ d = \frac{10 000}{\frac{30}{360} - 400 000} = 0,1 = 10\% $

\subsection{Przykład 32}
Załóżmy że wyztawca eksla z przykładu 31 ma możliwość otrzymania w dniu 5 maja trzymiesięcznej pożyczki w kwocie 390 000 PLN dzięki której mógłby zapłacić za towar i nie musiałby podpisywać weksla. Możemy wyznaczyć wysokość oprocentowania pożyczki przy której jej zaciągnięcie byłoby korzystniejsze od podpisywanie weksla. Zatem $d = 10\%, n = \frac{90}{360} $, więc (39) $ r = \frac{0,1}{1 - 0,1 \cdot \frac{90}{360}} = 10,26% $. Zatem pożyczka byłaby korzystniejsza, o ile jej oprocentowanie w skali roku byłaby niższa od 10,26 \%.


%
% Sekcja 15
%

\newpage

\section{Zasada równoważności kapitałów}
Wartość kapitału zmienia się w casie, We wszystkich rodzajach inwestycji podstawowe mają dwa pojęcia\\

- przyszła wartość kapitału - FV;\\

- obecna wartość kapitału - PV;\\

W poprzednich rozdziałach rozważaliśmy, w jaki sposób wyznaczyć obecną wartość kapitału,który mamy otrzymać lub zapłacić w przyszłości oraz przyszłą wartość kapitału, który posiadamy obecnie. Teraz rozszerzymy tę analizę na bardziej ogólne przypadki. Będziemy zakładać \textbf{złożoną kapitalizację odsetek}.\\

Aktualizacja wartości kapitału dotyczy kapitału, którego wartość jest znana dla ustalonego momentu i polega na obliczeniu jego wartości na inny moment (późńiejszy lub wcześniejszy). Aby zilustrować to pojęcie załóżmy, że wartość kapitału w chwili $ n_0 $ wynosi $ K(n_0) $, gdzie $ n_0 $ jest liczbą całkowitą. Wtedy korzystając ze wzorów (10) i (29), możemy wyznaczyć wartość tego kapitału w dowolnym momencie $ n $. Mianowicie mamy\\

\begin{center}
	\begin{equation}
		K(n) = K(n_0) \cdot (1 + r)^{n-n_0}
	\end{equation}
\end{center}\\

Wielkość $ K(n) $ nazywa się \textbf{zaktualizowaną wartością kapitału $ K(n_0) $ na moment $ n $}\\

Zasada rówmoważności kapitałów na dany moment: \textbf{kapitały $ K_1 $ i $ K_2 $ są rówoważne na moment $ n \epsilon Z $, jeżeli ich wartości zaktualizowane na moment $ n $ są równe}.\\

Rozważmy model opisany wzorem (40). Załóżmy że znane są wartości kapitałów $ K_1 $ i $ K_2 $ w dwóch ustalonych momentach $ n_1 $ i $ n_2 \epsilon Z $, tzn, są znane wielkości $ K_1(n_1) $ i $ K_2(n_2) $. Wtedy zgodnie ze wzorem (40) dla dowolnie ustalonego momentu $ n \epsilon Z $ zaktualizowanie wartości kapitałów $ K_1 $ i $ K_2 $ na ten moment wynoszą odpowiednio: \\

$ K_1(n) = K_1(n_1)(1 + r)^{n-n_1} $\\

oraz\\

$ K_2(n) = K_2(n_2)(1 + r)^{n-n_2} $\\

Zatem kapitały $ K_1 $ i $ K_2 $ są równoważne na moment $ n $ wtedy i tylko wtedy gdy zachodzi równość:\\

$ K_1(n_1)(1 + r)^{n-n_1} = K_2(n) = K_2(n_2)(1 + r)^{n-n_2} $\\

\begin{center}
	\begin{equation}
		K_1(n_1)(1+r)^{-n_1} = K_2(n_2)(1+r)^{-n_2}
	\end{equation}
\end{center}\\

Obserwacja prowadzi nas do następującego wniosku: \textbf{jeżeli dwa kapitały są równoważne na pewien moment, to są one równoważne na każdy moment.}\\

Uwzględniając ten fakt, możemy sformułować zasadę równoważności kapitałów w następujący sposób: \textbf{dwa kapitały są równoważne, jeżeli ich zaktualizowane wartości na jakikolwiek moment są równe}\\

\subsection{Przykład 33}
Zbadamy czy przy rocznej stopie procentowej równej 5\% kwota 10 000 PLN zainwestowana 2 lata temu jest równoważna kwocie 11 800 PLN, która będzie zainwestowana za rok.\\

Wystarczy sprawdzić czy spełniony jest warunek (41)\\

Ponieważ $ n_1 = -2, n_2 = 1, r=5\% $ zatem:\\

$ L = K_1(n_1)(1+r)^{-n_1} = 10 000(1 + 0,05)^2 = 11 025 $\\

$ P = K_2(n_2)(1+r)^{-n_2} = 11 800(1 + 0,05)^{-1} = 11 238,10 $\\

Zatem kapitały nie są równoważne.\\

\subsection{Przykład 34}
Przyjmując dane z poprzedniego przykładu. Wyznaczyć wartośc apitału, który zainwestowany za rok jest równoważny temu, który został zainwestowany przed dwoma laty. Korzystając z (41) otrzymujemy równość: $ 11 025 - K_2(n_2)(1+r)^{-n_2} $ stąd $ K_2(1) = 11 576,25  $\\

\subsection{Przykład 35}
W jakim monecie należy otrzymać kapisał 243 101,25 PLN, aby przy rocznej stopie procentowej r = 5\% był on równoważny kapitałowi 200 000 PLN uzyskanemu 3 lata temu.\\

$ n_1 = -3, K_1(-3) = 200 000, K_2(n2) = 243 101,25, r = 5\% $\\

Zatem warunek (41) prowadzi do równania:\\

$ 200 000(1 + 0,05)^3 = 243 101,25(1 + 0,05)^{-n_2} $\\

$ (1,05)^{n_2 + 3} = \frac{243 101,25}{200 000} $\\

$ (n_2 + 3) ln(1,05) = ln(\frac{243 101,25}{200 000}) $\\

$ n_2 + 3 = \frac{ln(1,05)}{ln(\frac{243 101,25}{200 000})} $\\

$ n_2 + 3 = 4 => n_2 = 1 $\\

Wykazaliśmy więc, że kapitały będą równoważne jeżeli pierwszy otrzymamy za rok.

\newpage

\section{Zasada równoważności przy kapitalizacji ciągłej}
Równoważność kapitałów można również badać przy założeniu kapitalizacji ciągłej. Wówczas odpowiednikiem warunku (41) jest następujący warunek:

\begin{center}
	\begin{equation}
		K_1(n_1) \cdot e ^{-r \cdot n_1} = K_2(n_2) \cdot e ^{-r \cdot n_2}
	\end{equation}
\end{center}\\

\subsection{Przykład 36}
Przy założeniu kapitalizacji ciągłej i rocznej stopie procentowej r = 5\% wyznaczymy taką wartośc kapitału, który mamy otrzymać za 4 lata aby był on równoważny kapitałowi o wartości 20 000 PLN, który mamy otrzymać za 2 lata.\\

$ n_q = 4, n_2 = 2, K_2(n_2) = 20 000, r = 5\%  $\\

ze wzoru (42)\\

$ K_1(n_1) \cdot e ^{-0,05\cdot 4} = 20 000 \cdot e6^{-0,05 \cdot 2} $\\

$ K_1(n_1)  = 20 000 \cdot e ^{0,1} = 22 103, 42 $

\newpage

\section{Stopa procenotwa a równoważność kapitału}

Odpowiedź na pytanie o równoważność dwóch kapitałów zależy od wartości rocznej stopy procentowej. Jeżęli przy ustalonej stopie procentowej dwa kapitały sa równoważne, to po jej zmianie przestają być równoważne. Obserwacaja taka wynika bezpośrednio z warunku (41)

\subsection{Przykład 37}

W przykładzie 34 stwierdziliśmy, że przy rocznej stopie procentowej równiej 5\%, kapitał o wartości 10 000 PLN zainwestowany przed dwoma laty jest równoważny kapitałowi o wartośći 11 576,25 PLN, który ma być zainwestowany za rok. Przypuśćmy, że wysokość rocznej stopy procentowej zmieniła się i wynosi $ r' = 4\% $. Wówczas\\

$ K_1(-2)(1+r')^{-(-2)} = 10 000 \cdot (1,04)^2 = 10 816 PLN $\\

$ K_1(1)(1+r')^{-1} = 11 576,25 \cdot (1,04)^{-1} = 11 131,01 PLN $\\

Zatem, po zmianie wysokości rocznej stopy procentowej, warunek (41) nie jest spełniony, wobec czego kapitały nie są równoważne.\\

Zauważmy, że mając dane wartości kapitałów w dwóch róznych mmentach i zakładając kapitalizację złożoną, na podstawie warunku równoważności kapitałów (41) moęmzy wyznaczyć wysokość rocznej stopy procentowej, przy której są równoważne. Istotnie, z (41) wynika, że\\

$ (1 + r)^{n_2 - n_1} = \frac{K_2(n_2)}{K_1(n_1)} $\\

Stąd\\

\begin{center}
	\begin{equation}
		r = (\frac{K_2(n_2)}{K_1(n_1)})^{\frac{1}{n_2-n_1}} - 1
	\end{equation}
\end{center}\\

Podobnie, zakładając kapitalizację ciągłą, na podstawie warunku rówoważności kapitalów (42), dostajemy\\

\begin{center}
	\begin{equation}
		r = \frac{1}{n_2 - n_1}ln(\frac{K_2(n_2)}{K_1(n_1)})
	\end{equation}
\end{center}\\

\subsection{Przykład 38 (kolos)}

Wyznaczymy wysokość rocznej stopy procentowej przy której kapitał 10 000 PLN zainwestowany przed 3 laty jest równowazny kapitałowi 16 000 PLN, który będzie zainwestowany za dwa lata.\\

Z warunku (41) zachodzi równość\\

$ 10 000(1 + r)^3 = 16 000(1 + r)^{-2} $\\

$ (1 + r)^5 = 1,6 $\\

$ r = (1,6)^{\frac{1}{5}} - 1 = 8,15 \% $\\

Wysokość rocznej stopy procentowej, przy tórej rozważane kapitały są równoważne, wynosi 8,15 \%.\\

\subsection{Przykład 39}

Zakładając kapitalizację ciągłą wyznaczymy wysokość rocznej stopy procentowej, przy której kapitał 15 000 PLN zainwestowany przed rokiem jest równoważny kapitałowi 19 800 PLN, który będzie zainwestowany za 5 lat.\\

$ n_1 = -1, n_2 = 5, K_1(n_1) = 15 000, K_2(n_2) = 19 800 $\\

Ze wzoru (44)\\

$ r = \frac{1}{5 - (-1)} ln (\frac{19 800}{15 000}) = \frac{1}{6} ln(\frac{198}{150}) = 4,63 \% $\\

\textbf{Uwaga} W przypadku kapitalizacji prostej dwa kapitały równoważne w jednym momencie mogą nie być równoważne w innym. Wobec trgo w tym przypadku nie istnieje pojęcie kapitałów równoważnych. 

\newpage

\section{Równoważność ciągów kapitałów}

Dotychczas rozważaliśmy szereg zagadnień związanych z równoważnością kapitałów. Analizę tę przeniesiemy teraz na ciągi kapitałów. Jak poprzednio, stale zakładamy złożoną kapitalizację odsetek. roczną stopę procentową oznaczamy przez r, zaś za jednostkę czasu przyjmiemy 1 rok.\\

Najpierw zajmiemy sie problemem równoważności kapitału i ciągu kapitałów. Rozważmy skończony ciąg kapitałów $ x_0, x_1, ..., x_n $. Każda z liczb $ x_i$, gdzie  $ i \epsilon {0,1,2, ..., n} $, może np. wyrażać nakład poniesiony przez inwestora w i-tym roku lub uzyskany przez niego w i-tym roku dochód. Nówimy, że dany kapitał jest \textbf{równoważny na mooment} $ k \epsilon Z $ ciągowi kapitałów $ x_0, x_1, ..., x_n $, jeżeli jego wartość zaktualizowana na moment k jest równa sumie zaktualizowanych na ten moment wartości wyrazów ciągu.\\

Ustalmy $ k \epsilon Z $ i załóżmy, że wartość kapitału w chwili $ n_0 \epsilon Z $ wynosi $ K_n_0 $. PRzez K_n_0(k) oznaczmy wartość tego kapitału zaktualizowaną na moment k. Niech ponadto $ x_i(k) $ dla $ i \epsilon {0, 1, 2, ..., n} $ oznacza wartość kampitału $ x_i $ zaktualizowaną na moment k. Wówczas rozważamy kapitał jest równoważny ciągowi kapitałów $ x_0, x_1, ..., x_n $ na moment $ k \epsilon Z $ wtedy i tylko wtedy, gdy\\

\begin{center}
	\begin{equation}
		K_n_0(k) = \Sigma ^n_{i=0}x_i(k)
	\end{equation}
\end{center}\\

Na podstawie wzoru (40) warunek (45) można zapisać w nastepującej równoważnej postaci 

\begin{center}
	\begin{equation}
		K_n_0 \cdot (1 + r)^{-n_0} = \Sigma ^n_{i=0}x_i \cdot (1 + r)^{-i}
	\end{equation}
\end{center}\\

Zatem japitał $K_n_0$ jest równoważny na moment $ k \epsilon Z $ ciągowi kapitałów $ x_0, x_1, ..., x_n $ wtedy i tylko wtedy, gdy spełniony jest warunek (46).\\

\textbf{Uwaga} Zauważmy, że jeżeli kapitał jest równoważny ciągowi kapitałów na pewien moment, to jest on równoważny temu ciągowi na każdy mooment.\\

W oparciu o powyższą uwagę, kapitał będziemy nazywać \textbf{równoważnym} ciągowi kapitałów $ x_0, x_1, ..., x_n $, jeżeli jest mu równoważny na jakikolwiek moment $ k \epsilon Z $.\\

\textbf{Uwaga} Mnożąc obie strony równoważności (46) przez $ (1+r)^{n_0} $, otrzymujemy 

\begin{center}
	\begin{equation}
		K_n_0 = \Sigma ^n_{i=0} x_i \cdot (1 + r)^{n_0 - i}
	\end{equation}
\end{center}\\

Wzór (47) pozwala wyznaczyć wartość, w chwili $ n_0 $ kapitału, który jest równoważny ciągowi kapitałów $ x_0, x_1, ..., x_n $. Przyjmując w (47) $ n_0 = 0 $ dostajemy w szcególności wzór na obecną wartość kapitału równoważnego ciągowi kapitałów $ x_0, x_1, ..., x_n $

\begin{center}
	\begin{equation}
		K_n_0 = \Sigma ^n_{i=0} \frac{x_i}{(1 + r)^i}
	\end{equation}
\end{center}\\

\subsection{Przykład 40}

Sprawdzimy czy orzy rocznej stopie procentowej równej 8\%, kapitał 120 000 PLN, który mamy otrzymać za 2 lata, jest równoważny na moment k = 3 następującemu ciągowi kapitalów:\\

$ x_0 = 10 000 PLN, x_1 = 5000 PLN, x_2 = 25000 PLN, x_3 = 30 000 PLN, x_4 = 20 000 PLN, x_5 = 35 000 PLN $\\

Ponieważ r = 8\%, mamy\\

$ \Sigma ^5_{i=0}\frac{x_i}{(1+r)^i} = 10 000 + \frac{5000}{1,8} + \frac{25000}{(1,08)^2} + \frac{30000}{(1,08)^3} + \frac{20000}{(10,8)^4} + \frac{35000}{(10,8)^5} = 98 399,07 $\\

Z drugiej strony, obecna wartość kapitału 120 000 PLN, który mamy otrzymać za 2 lata wynosi\\

$ K_n_0 = \frac{120 000}{(1,08)^2} = 102880,66 $ PLN\\

Zatem warunek (48) nie jest spełniony, czyli kapitał 120 000 PLN, który mamy otrzymać za 2 lata, nie jest równoważny rozważanemu ciągowi kapitałów.

\subsection{Przykład 41}

Przy założeniu, że roczna stopa procentowa jest równa 8\%, wyznaczymy kapitał, który otrzymamy za 2 alta jest równoważny ciągowi kapitałów z przykładu 40.\\

Stosując wzór (47) oraz korzystając z obliczeń z poprzedniego przykładu\\

$ K_2 = \Sigma ^5_{i=0} x_i \cdot (1+r)^{2-i} = (1,08)^2 \cdot \Sigma ^5_{i=0}\frac{x_i}{(1,08)^i} = 114772,68 $ PLN

\subsection{Przykład 42}

Sprawdzimy, czy przy rocznej stopie procentowej równej 5\%, kapitał 79 790, który mamy otrzymac za 3 lata, jest równnoważny nastepującemu ciągowi kapitałów:\\

$ x_0 = 20 000 PLN, x_1 = 15 000 PLN, x_2 = 12 000 PLN, x_3 = 27 500 PLN $\\

Wystarcyz sprawdzić czy zachodzi równość (47), $ n_0 = 3, r = 5\% $\\

$ \Sigma ^3_{i=0} x_i(1,05)^{3-i} = 79 790,00 = K_3 $\\

Wobec tego, zachodzi równość (47), czyli kapitał 79 90, który mamy otrzymać za 3 lata, jest równoważny podanemu ciągowi kapitałów.\\

Załóżmy, że mamy dwa ciągi kapitałów $ x_0, x_1, ..., x_n $ oraz $ y_0, y_1, ..., y_m $. Niech X będzie kapitałem równoważnym $ x_0, x_1, ..., x_n $, zaś Y niech będzie kapitałem równoważnym ciągowi $ y_0, y_1, ..., y_m $. Ciągi kapitałów nazywamy \textbf{równoważnymi} jeśli kapitały X i Y są równoważne. W przeciwnym padku mówimy, żę $ x_0, x_1, ..., x_n $ i $ y_0, y_1, ..., y_m $ są \textbf{nierównoważnymi} ciągami kapitałów.

\textbf{Uwaga} Przypomnijmy, że dwa kapitały są równoważne wtedy i tylko wtedy gdy ich obecne wartości są równe. Wobec tego na podstwie (48) otrzymujemy, że ciągi kapitałów $ x_0, x_1, ..., x_n $ i $ y_0, y_1, ..., y_m $ są równoważne wtedy i tylko wtedy gdy 

\begin{center}
	\begin{equation}
		\Sigma ^n_{i=0}\frac{x_i}{(1 + r)^i} = \Sigma ^m_{j=0}\frac{x_j}{(1 + r)^j}
	\end{equation}
\end{center}\\

\subsection{Przykład 43}

Przy założeniu, że roczna stopa procentowa r = 5\% sprawdzimy, czy podane ciągi kapitałów sa równowazne\\

$ x_0 = -4000 PLN, x_1 = 10000 PLN, x_2=8000 PLN, x_3 = 12000 PLN $\\

$ y_0 = 5000 PLM, y_1 = 2000 PLN, y_2 = 3000 PLM, y_4 = 7000 PLN, y_5 = 9000 PLN $\\

Mamy\\

$ \Sigma ^3_{i=0} \frac{x_i}{(1,05)^i} = 23 146,10 PLN $ \\

$ \Sigma ^4_{j=0} \frac{y_j}{(1,05)^j} = 23 077,04 PLN $ \\

Zatem równość (49) nie zachodzi, czyli ciągi nie sa równoważne.\\

\textbf{Uwaga} Równoważność ciągów kapitałów zależy od wartości rocznej stopy procentowej.\\

\textbf{Uwaga} Rówoważność ciągów kapitałów ma ścisły związek z badaniem efektywności inwestycji finansowych. Zagadnienem tym zajmiemy się szczegółowo w kolejnej części wykładu.\\

\newpage

\section{Mierniki oceny inwestycji finansowych}

Pojęcie inwestycji finanswoej jest na ogół ojarzone z zakupem akcji lub innych papierów wartościowych. W istocie ma ono jednak bardziej ogólne znaczenie i obejmuje szeroki zakres przedsięwzeięć podejmowanych z wykorzystaniem posiadanego kapitału. Każda inwestycjia finansowa wymaga nakładu, czyli zaangażowania pewnych środków finansowych, który daje prawo do ewentualnych dochodów w przyszłości. W działalności gospodarczej inwestycja finansowa najczęściej wiąże się z powieśzkeniem lub modernizacją środków trwałych.\\

Przez inwestycje finansową będziemy rozumieć ciąg płatności znany zarówno co do wielkości jak i momentów ich występowania. Płatności ujemna reprezentuje nakład inwestora, a dodatnia reprezentuje jego dochód. Jeżeli nakład i dochód występuje w tym samym momencie, to płatność w tym momencie jest suma tych dwóch wielkości. Stale będziemy zakładać żę pierwsza płatność jest nakładem (czyli jest ujemna), a moment jej wystąpienia jest początkiem okresu iwestycyjnego. Wśród pozostałych płatności co najmniej jedna powinna stanowaić dochód (czyli być dodatnia).\\

\textbf{Horyzontem czasowym} inwestycji nazywamy długość okresu objętego inwestycji. W całym wykładzie jednostkę czasu przyjmiemy 1 rok. Przez n oznaczamy horyzont inwestycyjny wyrażpny w latach.ś Z kolei prze $ x_j $ dla $ j \epsilon {0,1,2,..,n} $ oznaczać będziemy wysokość płatność w momencie j.

\subsection{Przykład 44}

śIwestycja wymagająca nakładów w wysokości 100 000 PLN obecnie i 50 000 PLN za rok, po drugim roku prz7niesie dochód w wysokości 40 000 PLN. Zaś po trzecim i czwartym roku 120 000 PLN.\\

W tym przypadku horyzont jest równy n = 4 lata. Ponadto płatności w kolejnych latach wynoszą: \\

$ x_0 = -100 000 PLN, x_1 = -50 000 PLN, x_2 = 40 000 PLN, x_3 = 120 000 PLN, x_4 = 120 000 PLN $\\

Istotnym problemem jest określenie celowości danej inwestycji finansowej. Służą do tego różne narzędzia, zwane miernikami oceny inwestycji finansowych. W tym wykładzie omówimy 3 najważniejsze z pośród nich:\\

Wartość bieżąca netto inwestycji\\

Wewnętrzna stopa zwrotu\\

Średni czas trwania\\

\newpage

\section{Wartość bieżąca netto}

Jedną z podstawowych miar służących do oceny decyzji inwestycji, jest \textbf{wartość bieżąca netto} ( w skrócie \textbf{NPV}). Jest to suma zdyskontoanych na moment 0 nakładów i dochodów z inwestycji przy ustalonej stopie procentowej. Przy założeniu kapitalizacji złożonej mamy:

\begin{center}
	\begin{equation}
		\NPV = \Sigma ^n_{j=0} \frac{x_j}{(1 + r)^j}
	\end{equation}
\end{center}\\

gdzie n jest czasem trwania inwestycji (w latach), $ x_j $ dla $ j \epsilon {0,1,..., n}} $ jest wartością płatności na koniec j-tego roku, zaś r oznacza roczną stopę procentową.

\subsection{Przykład 45}

Wyznaczymy wartość bieżącą netto inwestycji z przykładu 44. Przyjmiemy roczną stopę procentową r = 5\%.\\

Ponieważ n = 4, stosując (50), dostajemy\\

$ NPV = - 100 000 + \frac{-50 000}{1,05} + \frac{40 000}{1,05^2} + \frac{120 000}{1,05^3} + \frac{120 000}{10,5^4} = 91 046,94 PLN $\\

Do wyznaczania wartości bieżącej netto można zastosować wbudowaną formułę NPV dostępną w pakiecie Excel.

\subsection{Przykład 46}

Dla danych z przyładu 44 mamy NPV = -100 000 + NPV();\\

\textbf{Uwaga} Ze wzoru (50) wynika, że wysokość bieżąc netto inwestycji zależy od wysokości rocznej stopy procentowej. Fakt ten ilustruje kolejny przykład.

\subsection{Przykład 47}

Wyznaczymy wartość bieżącą netto inwestycji z przykładu 44, przy założeniu, że roczna stopa procentowa jest równa r = 6\%.\\

Na podstawie wzoru (50) otrzymujemy:\\

$ NPV = 84 235,60 PLN $\\

\textbf{Uwaga} Wartość bieżąca netto inwestycji ma następującą interpretacje. W porównaniu z rachunkiem bankowym oprocentowanym według stopy procentowej r, dana inwestycjia jest bardziej opłacalna, jeżeli jej wartośc bieżąca netto jest dodatnia. Jeżeli wartość bueżąca netto inwestycji jest ujemna, to inwestycja jest mniej opłacalna w porównaniu z rachunkiem bankowym oprocentowanym według rocznej stopy procenotwej r. Jeżeli natomiast wartość bieżąca  netto inwestycji jest równa zerom to inwestycja jest tak samo opłacalna jak lokata bankowa oprocentowana według rocznej stopy procentowej r.\\

\textbf{Uwaga} Wartość bieżąca netto inwestycji może y do porównania jej opłacalności nie tylko z lokatą bankową, lecz również z opłacalniścią innych inwestycji. Porównanie takie musi się jednak opierać na założeniu, że wartość bieżąca netto każdej z porównywanych inwestycji jest wyznaczona przy tej samej stopie procentowej. Zagadnienie to jest ściśle związane z pojęciem równoważności ciagów kapitałów, które omówiliśmy wcześniej.\\

Przypomnijmy ze ciągi kapitałów 

\begin{center}
	\begin{equation}
		\Sigma ^n_{i=0}\frac{x_i}{(1 + r)^i} = \Sigma ^m_{j=0}\frac{x_j}{(1 + r)^j} (nie pisać)
	\end{equation}
\end{center}\\

Wynika stąd że ciągi kapitałów są równoważne wtedy i tylko wtedy gdy ich wartości bieżące netto są równe.

\subsection{Przykład 48}
Inwestor ma do wyboru dwie możliwości inwestycji kapitału, przynoszące w kolejnych latach następujące płatności\\

inwestycja A: -5000, -10000, 0, 200000, 30000\\

inwestycja B: -5000, 10000, 10000, 10000, 5000\\

Sprawdzić czy inwestycje A i B sa równoważne przy rocznej stopie procentowej r = 5\%\\

Dla obydwu wyznaczymy NPV\\

A: $ NPV = 27 434,02 PLN $\\

B: $ NPV = 26 345,99 PLN $\\

Zatem NPV(B) < NPV(A), czyli inwestycje nie są równoważne przy rocznej stopie procentowej r = 5\%. Inwestycja A jest bardziej korzystna niż B.

\subsection{Przykład 49}

Rozważmy inwestycje z przykładu 48 założeniu że roczna stopa procentowa wynosi 7 \%. Wtedy wartości bieżące netto inwestycji wynoszą odpowiednio:\\

$ NPV(A) = 24 867,02 $\\

$ NPV(B) = 25 057,64 $\\

Wobec tego inwestycje również nie są równoważne, ale tym razem NPV(A) < NPV(B), czyli przy rocznej stopie procentwej r = 7\% inwesrycja B jest korzystniejsza od inwestycji A.

\textbf{Uwaga} Przykłady 48 i 49 pokazują zasadzniczą trudność związaną z ocena opłacalności poszczególnych inwestycji na podstawie ich wartości bieżącej netto. Ocena taka zależy bowiem od prawidłowego ustalenia wartości rocznej stopy procentowej.

\subsection{Przykład 50 (Kolos)}

Inwestor ma do wyboru dwie możliwości inwestycji kapitału, przynoszące w kolejnych latach następujące płatności\\

inwestycja A: $ -4000, -10000, 0, 20000, 40000 $\\

inwestycja B: $ -4000, 10000, 1000, 10000, W $\\

Dla jakiej wartości parametru W inwestycje A i B są równoważne przy rocznej stopie procentowej równej 4\%.\\

Dla obydwu inwestycji wyznaczamy bieżące wartości netto ciągów kapitałów\\

$ NPV(A) = 38 356,71 $\\

$ NPV(B) = .... + \frac{W}{(1 + 0,04)^4} = 23 750,91 + \frac{W}{(1 + 0,04)^4} $\\

$ 38 356,71 = 23 750,91 + \frac{W}{(1 + 0,04)^4} $\\

$ W = 17086,72 $\\

\newpage

\section{Wewnętrzna stopa zwrotu}

Drugim ważnym narzędziem służącym do oceny inwestycji finansowych jest \textbf{wewnętrzna stopu zwrotu (IRR)}. Jest to roczna stopa procentowa, dla której wartość bieżąca netto inwestycji jest równa 0. Oznaczając wewnętrzną stopę zwrotu przez r, na podstawie (50) dostajemy następujące równanie na r

\begin{center}
	\begin{equation}
		\Sigma ^n_{j=0}\frac{x_j}{(1 + r)^j} = 0
	\end{equation}
\end{center}\\

\textbf{Uwaga} Dla inwestycji o pojedynczym nakładzie wewnętrzna stopa zwrotu jest maksymalną stopą procentową przy której inwestycja jest opłacalna. Ściślej mówiąc jest to maksymalna stopa procentowa przy której inwestycja się zwraca.\\

\textbf{Uwaga} Zauważmy, że wyznaczanie wewnętrznej stopę zwrotu sprowadza się do wyznaczenia rozwiązań pewnego równania. Wobec tego, dla niektórych inwestycji może ona nie istnieć, zaś dla innych może nie być wyznaczona jednoznacznie. Można jednak wykazać, że dla każdej inwestycji w której ciąg dochodów poprzedzony jest ciągiem nakładów, wewnętrzna stopa zwrotu istnieje i jest, wyznaczona jednoznacznie. 

\subsection{Przykład 51 (kolos)}

Inwestycja wymagajaca nakładów w wysokości 300 000 obecnie i 100 000 za rok, po drugim roku przyniesie dochód w wysokości 50 000, zaś po trzecim i czwartym roku 250 000.\\

a) wyznaczyć wartość bieżącą netto tej inwestycji przy założeniu że roczna stopa procentowa jes równa 5\%.\\

b) Wyznaczyć wewnętrzną stopę zwrotu z tej inwestycji\\

a) $ NPV = 71 748,40 $\\

B) ze wzoru (52)\\

$ -300000 - \frac{100000}{1 + r} + \frac{50 000}{{1 + r}^2} + \frac{250 000}{(1 + r)^3} + \frac{250 000}{(1 + r)4} = 0 $\\

Skorzystamy z formuły IRR -> IRR(-300000;-100000;5000;250000;250000) = 10,81\% \\

\subsection{Przykład 52}

Rozważmy inwestycje o stępujących płatnościach: -4000, 5000, -2000.

Wewnętrzna stopa zwrotu dla tej inwestycji nie istnieje.

\newpage

\section{Średni czas trwania}

Rozważmy inwestycje finansową o oryzoncie czasowym n i płatnościach $ x_1, x_2, ..., x_n $ Niech r* będzie wewnętrzną stopą zwrotu z tej inwestycji. Wtedy zgodnie ze wzorem (52), mamy

	$	\Sigma ^n_{j=0}\frac{x_j}{(1 + r_*)^j} = 0 $

Przyjmiemy oznaczenie

\begin{center}
	\begin{equation}
		P_0 = \Sigma ^n_{j=1}\frac{x_j}{(1 + r_*)^j}
	\end{equation}
\end{center}\\

\textbf{Uwaga} Z (53) i (54) wynia, że

\begin{center}
	\begin{equation}
		P_0 = -x_0
	\end{equation}
\end{center}\\

\textbf{Średnim czasem trwania inwestycji} (ang. duration) nazywamy liczbę D określaną w następujacy sposób

\begin{center}
	\begin{equation}
		D = \frac{1}{P_0}\Sigma ^n_{j=1} \frac{x_j}{(a + r^*)^j} \cdot j
	\end{equation}
\end{center}\\

Średni czas trwania inwestycji jest zatem średnią ważoną momentów wystepowania płatności, przy czym wagami są zdyskontowane, przy założeniu wewnętrznej stopy zwrotu, udziały poszczególnych płatności w wartości bieżącej netto inwestycji. Przy wyborze inwestycji na podstwie średniego czasu trwania inwestor powinien kierować się jak najmniejszą wartością tego wskaźnika. \\

\subsection{Przykład 53}

Rozważmy inwestycje z przykładu 51. Wtedy r* = 10,81 \%. Pnadto $ P_0 = 300000 $, wobec tego stosując (55) otrzymujemy:

$ D = \frac{1}{300000} (-\frac{100000}{1 + 0,1081} \cdot 1 + \frac{50000}{(1 + 0,1081)^2} \cdot 2 + \frac{250000}{(1 + 0,1081)^3} \cdot 3 + \frac{250000}{(1 + 0,1081)^4} \cdot 4) = 4,02 $\\

Zatem średni czas trwania inwestycji wynosi 4,02 lat\\

\newpage

\section{Renty}

\textbf{Rentą} nazywamy ciąg płatności w równych odstępach czasu. Kolejne kwoty wypłacane z tytułu renty nazywamy \textbf{ratami renty}. Okres między dwiema kolejnymi ratami nazywa się \textbf{okresem bazowym}. Rentę o skończonej liczbie lat nazywa się \textbf{rentą okresową}, zaśrentę o nieskończonej liczbie lat - \textbf{rentą wieczystą}. rente której raty wypłacane są na koniec okresów bzowych nazyw asię \textbf{rentą płatną z dołu}. Jeżeli raty renty wypłacane są na początku okrexów bazowych to nzaywamy ją \textbf{rentą płatną z góry} Raty renty mogą być stałe albo zmieniać się w czasie. W dalszym ciągu ograniczymy się do rent o stałych ratach. Głównym celem naszych rozważań będzie zagadnienie \textbf{wyceny renty} tzn. wyznaczenie kapitału równoważnego rencie. Wycenę renty można oczywiście przeprowadzić na dowolny moment. W praktyce najważniejsze są wycena renty na jej początek, czyli wyznaczenie \textbf{początkowej wartości renty} oraz wycena renty ja jej koniec czyli wyznaczenie \textbf{końcowej wartości renty}.\\

- \textbf{Początkowa wartość renty} jest sumą wartości rat renty zaktualizowanycj na moment początkowy renty, \\

- \textbf{Końcowa wartośc renty} jest sumą wartości ran renty zaktualizowany na moment końcowy renty\\

Będziemy rozważać zarówno przypadek kapitalizacji rocznej, jak również model oparty na wielokrotnej kapitalizacji w ciągu roku.

\newpage

\section{Renty z roczną kapitalizacją odsetek}

Załóżmy, że z tytułu renty przez n lat wypłacana będzie corocznie \textbf{z dołu} ustalana kwota W. Przez r oznaczmy roczną stopę procentową. Wówczas początkowa wartość takiej renty, będąca sumą wartości jej rat zaktualizowanych na moment początkowy, jest równa: 

\begin{center}
	\begin{equation}
		P_D = W \Sigma ^n_{i=1}\frac{1}{(1+r)^i} (nie pracujemy na tym)
	\end{equation}
\end{center}\\

Stosując wzór na sumę n pozątkowych wyrazów ciągu geometrzycznego o pierwszych wyrazach.... dostajemy stad:

\begin{center}
	\begin{equation}
		P_D = \frac{W}{r}(1 - \frac{1}{(1 + r)^n})
	\end{equation}
\end{center}\\

Z kolei wartość końcowa rozważanej renty jest równa,

\begin{center}
	\begin{equation}
		F_D = P_D(1 + r)^n
	\end{equation}
\end{center}\\

\begin{center}
	\begin{equation}
		F_D = \frac{W}{r}((1 + r)^n - 1)
	\end{equation}
\end{center}\\

Oznaczając przez $ P_G $ początkową wartośc renty płatnej \textbf{z góry} mamy

\begin{center}
	\begin{equation}
		P_G = W \Sigma ^{n-1} (aaa nie warto uzuwac tego)
	\end{equation}
\end{center}\\ 

\begin{center}
	\begin{equation}
		P_G = W\frac{1 + r}{r}(1 - \frac{1}{(1 + r)^n})
	\end{equation}
\end{center}\\ 

Wartość końcowa takiej renty wynosi

\begin{center}
	\begin{equation}
		tylko do liczenia
	\end{equation}
\end{center}\\ 

Ze wzoru (60) i (62) otrzymujemy:

\begin{center}
	\begin{equation}
		F_G = P_G(1 + r)^n
	\end{equation}
\end{center}\\ 

Zatem uwzgledniając (61) otrzymujemy

\begin{center}
	\begin{equation}
		F_G = W\frac{1 + r}{r} \cdot ((1 + r)^n - 1)
	\end{equation}
\end{center}\\ 

\textbf{Uwaga} Ze wzorów (57) i (58) otrzymujemy naatępujący związek między początkowymi wartościami renty płatnej z dołu i renty płatnej z góry

\begin{center}
	\begin{equation}
		P_G = P_D(1 + r)
	\end{equation}
\end{center}\\ 

\subsection{Przykład 54}

Przyjmując roczną stopę procentową równą 5\% i zakładając roczną kapitalizację odsetek, wyznaczyć początkową wartość 8-letniej renty płatnej corocznie a) z dołu, b) z góry w kwocie 10 000 PLN. Ile wynosi końcowa wartość takiej renty?\\

\subsubsection{a)}

Korzystając ze wzoru (57)\\

$ P_D = 64 632,13 PLN $\\

Ponadto, na podstawie wzoru (58), mamy\\

$ F_D = 95 491,09 PLN $\\

\subsubsection{b)}

Stosując wzór (65) i korzystając z obliczeń z wykonanych w punkcie a

$ P_G = 67863,74 PLN $\\

Uwzględniając (63) mamy\\

$ F_G = 100 265,65 PLN $\\

\newpage

\section{Renty z wielokrotną kapitalizacją odsetek w ciągu roku}

Załóżmy, że kapitalizacja odsetek odbywa się m-krotnine w ciągu roku (w równych odstępach czasu). Przyjmijmy, że raty renty w wysokości W wypłacane są m razy w ciągu roku przez okres l lat i n spośród m podokresów l + 1 roku ($ 0 <= n < m $). Niech r oznacza nominalną stopę procentową. Wtedy początkowa wartość renty płatnej z dołu, tzn. na końcu każdego podokresu, wynosi

\begin{center}
	\begin{equation}
		P^{(m)}_D = \frac{W}{\frac{r}{m}} \cdot (1 - \frac{1}{(1 + \frac{r}{m})^{lm + n}}) 
	\end{equation}
\end{center}\\

W przypadku renty płatnej z góry

\begin{center}
	\begin{equation}
		P^{(m)}_G = W \cdot \frac{1 + \frac{r}{m}}{\frac{r}{m}} \cdot (1 - \frac{1}{(1 + \frac{r}{m})^{lm + n}}) 
	\end{equation}
\end{center}\\

\subsection{Przykład 55}

Przyjmując nominalną stopę procentową równą 4\% i zakładając kwartalną kapitalizację odsetek wyznaczymy początkową wartość renty płatnej co kwartał: a) z dołu, b) z góry w wysokości 2 500 PLN przez 3 lata i 3 miesiące.\\

Stosując wzory (66) i (67) otrzymujemy\\

\subsubsection{a)}

$ P^{(4)}_D = 30 334,35 PLN $\\

\subsubsection{b)}

$ P^{(4)}_G = 30 637,69 PLN $\\

\textbf{Uwaga} W praktyce często mamy doczynienia z rentami o zmiennych ratach. W takim przypadju jeżeli raty renty płatnej z dołu wynosza w koeljnych latach $ W_1,...., W_n $, to jej wartość początkowa jest równa

\begin{center}
	\begin{equation}
		P_D = \Sigma ^n_{i=1} \frac{W_i}{(1 + r)^i}
	\end{equation}
\end{center}\\

zaś wartość końcowa jest równa\\

\begin{center}
	\begin{equation}
		F_D = \Sigma ^n_{i=1} \frac{W_i}{(1 + r)^{n-i}}
	\end{equation}
\end{center}\\

W przypadku renty płatnej z góry mamy 

\begin{center}
	\begin{equation}
		P_D = \Sigma ^{n-1}_{i=0} \frac{W_i}{(1 + r)^i}
	\end{equation}
\end{center}\\

oraz\\

\begin{center}
	\begin{equation}
		F_D = \Sigma ^{n-1}_{i=0} \frac{W_i}{(1 + r)^{n-i}}
	\end{equation}
\end{center}\\

\newpage 

\section{Wyznaczanie wartości końcowej i wartości początkowej w arkuszu MS Excel}

Do wyznaczani wartości końcowej można zastosować wbudowaną formułę \textbf{FV}. Jej argumentami są: stopa procentowa, liczba rat, wysokość raty, saldo początkowe i typ. Argument typ dotyczy momentu płatności i wynosi: 0 dla rat płatnych z dołu, zaś 1 dla rat płatnych z góry.\\

\textbf{FV(stopa procentowa; liczba rat; rata; saldo początkowe; 0 lub 1}\\

Dwa ostatnie argumenty można pominąć. Wtedy dla salda poczatkowego zostanie przyjęta domyślna wartość 0. Dla argumentu typ wartością domyślną również jest 0 tzn. uzyskany w ten sposób wynik dotyczy rat płatnych z dołu.\\

\subsection{Przykład 56}

Na rachunku bankowym zdeponowano 20 000 PLN. Roczna stopa procentowa jest równa 5\%. Wyznaczymy saldo rachunku po 3 latach, jeżeli na początku każdego kwartału wypłacana będzie z niego kwota 800 PLN.\\

Mamy\\

FV(5\%/4;12;800;-20000;1) = 12 798,20 PLN\\

\subsection{Przykład 57}

Obecne zadłużenie wynosi 100 000 PLN. Obliczymy poziom zadłużenia po 5 latach, jeżeli przy rocznej stopie procentowej równej 4\% na koniec kazdego miesiąca spłacana będzie rata w wysokości 1 000 PLN.\\

FV(4\%/12;60;1000;-100000;0) = 55 800,68 PLN -> Zatem zadłużenie po 3 latach będzie wynisiło 55 800,68

Do wyznaczania wartości początkowej można zastosować wdubowaną formułę \textbf{PV}. Formuła posiada te same argumenty co formuła FV oraz tą samą konwencję.\\

\textbf{PV(stopa procentowa; liczba rat; rata; saldo początkowe; 0 lub 1}\\

\subsection{Przykład 58}

Przy założeniu, że roczna stopa procentowa jest równa 5\% wyznaczymy poczatkową wartość 24 rat płaconych w wysokości 500 PLN na koniec kolejnych miesięcy.\\

Mamy\\

PV(5\%/12;24;-500;0;0) = 11 396,95 PLN

\subsection{Przykład 59}

Przy założeniu, że roczna stopa procentowa jest równa 5\% wyznaczymy początkową wartość 15 rat płaconych w wysokości 800 PLN na początku kolejnych kwartałów.\\

PV(5\%/4;15;-800;0;1) = 11 016,44 PLN

\subsection{Przykład 60}

Przy założeniu że roczna stopa procentowa jest równa 6\%, wyznaczymy kwotę kredytu spłacanego w 12 kwartalnych ratach płatnych z dołu w wysokości 2 000 PLN.\\

PV(6\%/4;12;-2000;0;0) = 21 815,01 PLN

\newpage	

\section{Spłata rat kredytu}

\textbf{Wprowadzenie}\\

Przeprowadzimy teraz analizę spłaty rat kredytu. Udzielenie kredytu jest szczególnym przypadkiem inwestycji finansowej. Inwestorem jest strona udzielająca kredytu, zaś raty spłaty długu stanowią ciąg zwrotów z inwestycji. Takie spojrzenie na kredyt jest zgodne z praktyką. Instytucja finansowa lub osoba fizyczna podejmująca decyzję o przeznaczeniu środków na udzielenie kredytu pozbawia się innych możliwości ich zainwestowania. Prezentowane poniżej metody analizy dotyczace ratalnej spłaty kredtu pierają się na oprocentowaniu złożonym, na pojęciu wartości kapitału w czasie oraz na zasadzie równoważności ciagów kapitałów. 

\section{Zasada równoważności długu i rat}

Załóżmy, że w momencie n = 0 zaciągnięty zostal kredyt w wysokości $ K_0 $. Przyjmijmy, ze kredyt będzie spłacany w równych odstępach czasu, w n ratach o wartościach $ R_1, R_2, ..., R_n $. Kwota każdej raty zawiera zwrot części kapitału wraz z odsetkami, ale nie obejmuje kosztów takich jak np. prowizja czy ubezpieczenie. Analiza ratalnej spłaty kredytu opiera się na nastepującej zasadzie równoważności długu i rat. \\

Kredyt o wartości $ K_0 $ jest \textbf{równoważny} na moment n = 0 ciągowi rat $ R_1, ..., R_n $ płatnych w momentach i = 1, ... n, jeżeli kapitały przekazane sobie nawzajem przez wierzyciela i dłużnika są równoważne na moment 0.

Przy założeniu kapitalizacji złożonej zasada równoważności długu i rat prowadzi do warunku

\begin{center}
	\begin{equation}
		K_0 = \Sigma ^n_{i=1} \frac{R_i}{(1 + r_{okr})^i}
	\end{equation}
\end{center}\\

gdzie $ r_{okr} $ jest podokresową stopą procentową. Mnożoąc obie strony równania (72) przez $ \frac{R_i}{(1 + r_{okr})^n} $. dostajemy warunek równowaśności długu i rat na moment n.

\begin{center}
	\begin{equation}
		K_0(1 + r_{okr})^n = \Sigma ^n_{i=1}R_i(1 + r_{okr})^{n-i}
	\end{equation}
\end{center}\\

Jeżeli wszystkie raty kredytu są równe:

to warunek (72) przymuje postać

\begin{center}
	\begin{equation}
		K_0 = r \cdot \Sigma ^n_{i=1} \frac{1}{(1 + r_{okr})^i}
	\end{equation}
\end{center}\\

Stosując wzór na sumę wyarzów szeregu geometrzycznego, dostajemy zatem

\begin{center}
	\begin{equation}
		K_0 = \frac{R}{r_{okr}}(1 - \frac{1}{(1 + r_{okr})^n})
	\end{equation}
\end{center}\\

Wyznaczając stąd R otrzymujemy:

\begin{center}
	\begin{equation}
		R = \frac{K_0 \cdot r_{okr}}{1 - \frac{1}{(1 + r_{okr})^n}}
	\end{equation}
\end{center}\\

\subsection{Przykład 61}

Kredyt w kwocie 40 000 PLN zaciągnięty na okres 5 lat ma być spłacany w równych ratach: a) rocznych, b) miesięcznych. Przy założeniu, że nominalna stopa procentowa jest równa 4\%, wyznaczymy wysokość raty.

\subsubsection{a)}

ze wzoru (75) $ K_0 = 40 000, n = 5 i r_{okr} = r = 4\% $\\

$ R = (\frac{40 000 \cdot 0,04}{(1 - \frac{1}{1,04^5})}) = 8985,08 PLN $

\subsubsection{b)}

Najpierw wyznaczymy miesięczną stopę procentową równoważną rocznej stopie procentowej $ r = 4\% $.\\

$ r_{okr} = 1,04^{\frac{1}{12}} - 1 = 0,3274% $ \\

Zatem, stosując (75) mamy $ K_0 = 40000, n = 60, r_{okr} = 0,3274 $\\

$ R = 735,38 PLN $

\subsection{Przykład 62}

Pięcioletni kredyt ma być spłacany w rocznych ratach płatnych na koniec kolejnych lat w wysokości: 6 000PLN po pierwszym i drugim roku, 7 200 PLN po trzecim roku, 8 000 PLN po czwartym roku i 9 000 PLN po piątym roku. Przyjmując, że roczna stopa procentowa wynosi 8\% wyznaczymy kwotę kredytu.\\

Stosując wzór (72), n = 5, $ r_{okr} = r = 8\%, R_1 = R_2 = 6000, R_3 = 7200, R_4 = 8000, R_5 = 9000 $\\

$ K_0 = \frac{6000}{1,08} + \frac{6000}{(1,08)^2} + \frac{72000}{(1,08)^3} + \frac{8000}{(1,08)^4} + \frac{9000}{(1,08)^5} = 28420,67PLN $\\

\subsection{Przykład 63}

Czteroletni kredyt ma być spłacany w kwartalnych ratach płatnych w następującej wysokości: 1 PLN na koniec pierwszych 5 kwartałów, 2 000 PLN na koniec kolejnych pięciu kwartałów i 2 500 PLN na koniec każdego z pozostałych kwartałów. Wiedząc, że roczna stopa procentowa wynosi 8\%, wyznaczyć kwotę kredytu.\\

Rozpoczniemy od wyznaczenia kwartalnej stopy: $ r_{okr} = 1,08^{\frac{1}{4}} = 1,9427% $\\

Zatem ze wzoru (72), n = 16, $ r_{okr} = 1,9427% $, $ R_1,..R_5 = 1000, R_6,...R_10 = 2000, R_11,...R_16 = 2500 $\\

$ K_0 = 24 872,76 $\\

\newpage

\section{Schematy spłaty długu}

Omówimy teraz zagadnienia związane  zzaldem zadłużenia w dowolnym momencie spłaty kredytu. Zauważmy najpierw, że mnożąc obie strony równości (72) przez $ (1 + r_{okr})^j $ gdzie $ j \epsilon (1,...,n) $, dostajemy aktualizację długu i wartości poszczególnych rat na moment j. Przedstawiając prawa stronę tej równości w postaci sumy rat już zapłaconych do momenty j (włącznie) i rat, które pozostały jeszcze do splacenia trzymujemy:

\begin{center}
	\begin{equation}
		K_0(1 + r_{okr})^j = \Sigma ^j_{i=1}R_j(1 + r_{okr})^{j-i} + \Sigma ^n_{i=j-i}R_j(1 + r_{okr})^{j-i}
	\end{equation}
\end{center}\\

\textbf{Uwaga} Z punktu widzenia wierzyciela rówośc (76) pozwala na stwierdzenie, jaka wartość pożyczonego kapitału została odzyskana do momentu j, a jaka pozostała jeszcze do odzyskania, Z kolei, z punktu widzenia dłużnika, równość (76) informuje o tym, jaka wartość otrzymanego kapitału wraz z naliczonymi odsetkami została już oddana, a jaka pozostaje do oddania

\textbf{Długiem bieżącym} $ K_j $ w momencie $ j \epsilon {0,,...n} $ nazywa się wartość kapitału pozostałego do spłacenia po zapłacenia raty $ R_j $. Z równości (76):


\begin{center}
	\begin{equation}

	\end{equation}
\end{center}\\

\begin{center}
	\begin{equation}
		K_j = \Sigma ^n_{i=j+1} R_i(1 + r_{okr})^{j-1}
	\end{equation}
\end{center}\\

\textbf{Uwaga} We wzorach (77) i (78) przyjmujemy konwenscję, opdowiednio\\

$ \Sigma ^0_{i=1} R_i()... = 0 oraz \Sigma ^n_{i=n+1}... = 0 $

\textbf{Uwaga} Wzór (77) wyraża zależność długu bieżącego od długu początkowe i rat już zapłaconych, zaś wzór (78) wyraża zależność długu bieżącego od rat, które jeszcze nie zostały spłacone. Pierwsza z tych zalezności nazywa się \textbf{zależnością retrospetywną},  druga \textbf{zależnością prospektywną}. Zarówno z zależności retro czu prospektywnej wynika, że dług bieżący w momencie j = 0 wynosi $ K_0 $ , zaś dług bieżący w momencie j = n, czyli po zapłaceniu ostatniej raty wynosi 0

\textbf{Uwaga} Wartość długu bieżącego w danym momencie spłaty kredytu jest ważną informacją zarówno dla wierzyciela jak i dłużnika. Można ona w szcególności stanowić podstawę do zmiany wartości przyszłych rat, np. z powodu zmiany wysokości stopy procentowej lub z powodu restrukturyzacji kredytu.\\

Dla każdego $ j \epsilo {1,...,n} $ przez $ T_j $ oznaczmy część kwoty pożyczki spłaconą w j-tej racje, przez $ Z_j $ odsetki spłaconej w j-tej racie, zaś $ K_j $ resztę długu pozsotałą do spłacenia po spłaceniu j-tej raty. Dla każdego $ j \epsilon {1,...n} $: \\

- wielkość $ T_j $ nazywa się \textbf{częścią kapitałową} j-tej raty;\\
- wielkość $ Z_j $ nazywa się \textbf{częścią odsetkową} j-tej raty;\\

Zauważmy że \\

$ R_j = T_j + Z_j $\\

Dla każdego $ j \epsilo {1,...,n} $, odsetki spłacone w j-tej racie są wyznaczone według okresowej stopy procentowej $ r_{okr} $ na podstawie stanu zadłużenia na poczatku j-tego okresu. Wynika stąd, że:

\begin{center}
	\begin{equation}
		Z_j = K_{j-1} \cdot r_{okr}, dla j \epsilon{1,2,...,n}
	\end{equation}
\end{center}\\

Z kolei kwota długu spłacenia w j-tej racie wynosi

\begin{center}
	\begin{equation}
		T_j = R_j - Z_j, dla j \epsilon{1,2,...,n}
	\end{equation}
\end{center}\\

\textbf{Uwaga} Na podstawie zależności prospektywnej (78), dla każdego $ j \epsilon{1,2,...,n} $ mamy $ K_j = K_{j-1} + K_{j-1} \cdod r_{okr} - R_j $, uwzgledniając (79) dostajemy stąd

\begin{center}
	\begin{equation}
		K_j = K_{j-1} - T_j, dla j \epsilon{1,2,...,n}
	\end{equation}
\end{center}\\

\textbf{Uwaga} Warto podkreślić, że część odsetkowa raty $ Z_j $ jest zdefiniowana jako wartość odsetek należnych za j-ty okres, a nie jako wartość odsetek spłacanych w tym okresie. W niektórych okresach rata może być niższa niż naliczone odsetki, w związku z czym nie mogą być one spłacone. Ponadto, gdyby w danej racie spłacony był jedynie kapitał, a odsetki nie, to dłużnik spłaciłby część kredytu, a jednocześnie zaciągnąłby nowy kredyt, o wartości równej odsetkom, które nie zostały zapłacone. Aby uniknąć takiej sytuacji wygodniej jest założyć, że \\

- jezeli $ R_j >= Z_j $ dla pewnego $ j \epsilon {1,...n} $ to w j-tej racie spłacane są odsetki $ Z_j $ a dług zmniejsza się o $ T_j = R_j - Z_j $\\
- jeżeli $ R_j < Z_j $ dla pewnego $ j \epsilon {1,...n} $ to w j-tej racie spłacane są odsetki $ Z_j $ a dług zwiększa się o $ Z_j - R_j $\\

Przestawione założenia nazywane sa \textbf{priorytetem spłaty odsetek}\\

\textbf{Uwaga} Rozkład raty na część kapitałową i odsetkową pozwala prześledzić proces umarzania bieżących odsetek i długu przez kolejne raty. Do opisu tego procesu stosuje się tabelę zwaną \textbf{schematem spłaty długu}. Jej wiersze dotyczą koe=lejnych okresów spłaty długu, zaś w jej kolumnach znajdują się kolejno:\\

- j - numer okresu bazowego\\
- $ K_{j-1} $ - dług bieżący na poczatku j-tego okresu bazowego\\
- $ R_j $ - rata spłacana w j-tym okresie bazowym\\
- $ Z_j $ - część odsetkowa raty $ R_j $\\
- $ T_j $ - część kapitałowa raty $ R_j $\\
- $ K_j $ - dług bieżący na koniec j-tego okresu bazowego\\

\textbf{Uwaga} Przedstawimy teraz 2 przykłady dotyczące schematów spłaty kredytu w równych ratach oraz kredytu o równych częściach kapitałowych. Warto wspomnieć, że innymi wystepującymi w praktyce schematami spłaty kredytów są spłata odsetek w jednej racie i równe raty kapitałowe oraz bieżąca spłata odsetek i zwrot kapitału w ostatniej racie.

\subsection{Przykład 64}

Zbudować schemat spłaty kredytu z przykładu 61a). Przypomnijmy, że rata wynosi 8 985,08

\subsection{Przykład 65}

Jak wspomnieliśmy wczesniej, poza kredytami o stałych ratach, często stosowanym rodzajem kredytów są kredyty o stałych częściach kapitałowych. Pokażemy schemat konstrukcji spłaty takiego kredytu. rozważać będziemy kredyt z przykładu 64. Zauważmy najpierw, że: $ T_j = \frac{40000}{5} = 8000PLN dla j \epsilon {1,2,3,4,5} $

\section{Rzeszywista stopa opocentowania (RRSO)}

Wcześniej zauważylismy, że udzielenie kredytu można w naturalny sposób traktować jako inwestycję finansową. Wobec tego do jej oceny należy używać mierników stosowanych w analizie efektywności inwestycji. Kilka z nich omówiliśmy na poprzednich przykladach.Do najważniejszych  mierników należy wewnętrzna stopa zwrotu (IRR), która w kontekście udzielania kredytu nazywa się rzeczywistą stioą oprocentowania. Opiera się ona na zasadzie równoważności długu i rat,\\

Załóżmy, że kredyt w kwocie $ K_0 $ jest spłacany przez wieczyciela w n ratach równych $ R_1, ...R_n $ płatnych w wyrażonych w latach chwilach $T_1, ...,t_n $ od momentu t = 0 w którym kredyt został udzielony\\

\textbf{Rzeczysitą roczną stopą oprocentowania RRSO} nazywamy roczną stopę procentową, która jest rozwiązaniem równania.

\begin{center}
		$ K_0 = \Sigma ^n_{i=1} R_i(1 + r)^{-t_i} $
\end{center}\\

Przyjmijmy, że istnieje jednostka czasu zwana okresem bazowym za pomoca której termin dowolnej raty kredytu można wyrażyć liczbą całkowitą. W praktyce założenie te jest spełnione, gdyż na ogół terminy spłat ustalane sa na koniec lat, kwartałów, czy miesięcy. W szczególności bazowy okres można przyjąć 1 dzień. Załóżmy, że rok składa się z m okresów bazowych, wtedy dla każdego $ i \epsilon {1,2,...,n} $\\

\begin{center}
	\begin{equation}
		K_0 = \Sigma ^n_{i=1} R_i(1 + r)^{\frac{k_i}{m}}
	\end{equation}
\end{center}\\

Niech $ r_{okr} $ będzie stopą podokresową równoważną rocznej stopie r. Wtedy

\begin{center}
	\begin{equation}
		(1 + r_{okr})^m = 1 + r
	\end{equation}
\end{center}\\

stąd wynika, że $ (1 + r)^{\frac{1}{m}} = 1 + r_{okr} $

\begin{center}
	\begin{equation}
		K_0 = \Sigma ^{K_n}_{j=1} R_j(1 + r_{okr})^{-j}
	\end{equation}
\end{center}\\

Stopa proentowa $ r_{okr} $ która jest rozwiązaniem równania (85) nazywa się \textbf{okresową strzeczystą stopą oprocenowania}\\

\textbf{Uwaga} Z równości (84) wynika, że okresowa rzeczysita stopa oprocentowania jest równoważna rocznej rzeczysitej stopie oporcenotwania. \textbf{RRSO jest równoważnie ORSO}\\

\textbf{Uwaga} W celu wyznaczenia rzeczywistej rocznej stopy oprocentowania należy rozwiązać równanie (82) ze względu na r. Podobnie, zeby wuznaczyć okresową rzeczysitą stop∑ oprocentowania, należy roziwązać równanie (85) ze względu na $ r_{okr} $. Można się posłużyć Excelem. Rzeczywistą stopę oprocentowania można wyznaczyć przy użyciu formuły \textbf{IRR}. Jeżeli raty są równe to do wyznaczenia rzeczysitej rocznej stopy oprocentowania można zastosować formułę \textbf{RATE}. Jej argumentami są liczba rat; wysokośc rat; wysokośc kredytu ; saldo początkowe ; typ.\\

\subsection{Przykład 66}

Wyznaczyć rzeczywistą roczną stopę oprocentowania kredytu w wysokości 100 000 PLN słacanego a) na początku b) na końcu kolejnych pięcii lat ratami w wysokości: 20 000PLN w pierwszym i drugim roku oraz 30 000 PLN w 3,4,5 roku.\\

Mamy\\

\subsubsection{a)}

RRSO = IRR(-80000;20000;30000;30000;30000) = 13,22 \%

\subsubsection{b)}

RRSO = IRR(-100000;20000;20000;30000;30000;30000) = 8,68 \%

\subsection{Przykład 67}

Sześcioletni kredyt w kwocie 12 000 PLN jest spłacony w równych rocznych ratach płatnych z dołu w wysokości 2 400 PLN.\\

IRR = 5,47 \%\\

RATE = (6 ; -2400 ; 12 000)\\

\subsection{Przykład 68}

Rozważmy kredyt z przykładu 67, ale przy założeniu że raty płacone są z góry.\\

IRR = 7,93 \% \\

RATE = (5 ; -2400 ; 9600)\\

\subsection{Przykład 69}

Kredyt w wysokości 10 000 PLN będzie spłacony przez 2 lata w równych miesięcznych ratach. płatnych w wysokości 500 PLN na koniec kolejnych miesięcy. Wyznaczyć rzeczywistą roczną stopę oprocentowania tego kredytu.\\

Stosująć RATE dostajemy wartość okresowej rzeczywistej stopy oprocentowania. \\

RATE(24;-500;10000) = $ r_{okr} $ = 1,5131 \% \\

RRSO jest roczną stopą oprocentowania równoważną stopie $ r_{okr} $ Mamy zatem: $ RRSO = (1 + r_{okr})^{12} - 1 = (1,015131)^{12} - 1 = 19,79 \% $

\subsection{Przykład 70}

Kredyt w wysokości 6 000 PLN będzie spłacany przez rok w miesięcznych ratach płatnych w wysokości 550 PLN na koniec pierwszych 4 miesięcy, 525 PLN na koniec kolejnych 4 miesięcy i 500 PLN na konie ostatnich 4 miesięcy, Wyznaczyć RRSO.\\

$ r_{okr} = IRR(-6000;550;550;550;550;525;525;525;525;500;500;500;500) = 0,7741 \% $\\

$ RRSO = (1 + 1,7741)^{12} - 1 = 9,70 \% $

\newpage

\section{Wyznaczanie wysokości raty kredytu o równych ratach w arkuszu kalkulacyjnym Excel}

Do wyznacznia wysokości raty kredytu o równych ratach można zastosować wbudowaną formułę \textbf{PMT}\\

\textbf{PMT(stopa procentowa ; liczba rat ; wysokość kredytu ; saldo początkowe ; 0 lub 1)}\\

\subsection{Przykład 71}
Przy zaŻiżeniu, że nominalna stopa procentowa równa jest 6\% wyznaczyć wysokość raty kredytu w wysokości 100 000 PLN zaciągniętego na okres 5 lat, spacanego w miesięcznych ratach płatnych a> z dołu, b) z góry\\

\newpage

\section{Wyznaczanie części odsetkowej i części kapitałowej rat kredytu o równych ratach w arkuczu Excel}

Do wyznaczenia cześci odsetkowej rat kredytu o równych ratach można zastosować wbudowaną formułę \textbf{IPMT}. Jej argumentami są: stopa procentowa, numer raty, liczba rat, wysokość kredytu, saldo początkowe i typ.\\

\textbf{IPMT(stopa  procentowa ; numer raty ; liczba rat ; wysokość kredytu ; saldo początkowe ; 0 lub 1)}\\

Do wyznaczenia części kapitałowej rat kredytu o równych ratach można zastosować formułę \textbf{PPMT}. Jej argumenty sa identyczne jak w IPMT.\\

\subsection{Przykład 72}
Przy założeniu, że nominalna stopa procentowa jest rowna 8\%, wyznaczyć wartość raty kredytu w wysokości 15 000 PLN zaciągniętego na okres 2 lat i spłacanego w równych miesięcznych ratach płatnych z dołu.Następnie wyznaczyć wartość części odsetkowej, części kapitałowej dziewiątej raty tego kredytu.\\

PMT(8\%/12 ; 24 ; -15000) = 678,41\\

IPMT(8\%/12 ; 9 ; 24 ; -15000) = 68,42\\

PPMT(8\%/12 ; 9 ; 24 ; -15000) = 609,99\\

\newpage

\section{Wyznaczanie rat kredytu przy zmianie oprocentowania}

Załóżmy, że w trakcie spłaty kredytu stopa procentowa uległa zmianie. Wówczas zachodzi konieczność wyznaczenia nowej wysokości pozostałych do spłaty rat kredytu. \\

\subsection{Przykład 73}

Kredyt w wysokości 50 000 PLN zaciągnięty na okres 4 lat był spłacany równymi ratami, płatnymi na koniec kolejnych kwartałów. Nominalna stopa procentowa, która początkowa była równa 5\%, od początku czwartego roku wzrosła do 6\%. Wyznaczyć wysokość pozostałych do spłaty rat kredytu.\\

1. Najpierw obliczymy wysokośc raty przed podniesieniem stopy procentowej. Mamy: PMT(5\%/4 ; 16 ; -50000) = 3 467,34 PLN\\

2. Następnie obliczamy, do jakiego poziomu zmaleje dług po trzech latach splaty rat.Stosujać FV lub wzór: FV(5\%/4 ; 12 ; 3467,34 ; -50000) = 13 446,53\

3. Na kkoniec wyznaczymy wysokość rat po zianie stopy proentowej. W tym celu zastosujemy formułę PMT: PMT(6\%/4 ; 4 ; -13466,53) = 3488,63\\

\newpage

\section{Tablice trwania życia}

\section{Przyszły czas życia}

\textbf{Przyszły czas życia} x-latka, tzn. czas który pozostał mu do smierci, będziemy oznaczać przez $ T_x $. Oczywiście $ T_x $ na ogół nie jest dokładnie znana. Wobec tego bedziemy traktować $ T_x $ jako zmienną losową. Przyjmuje ona wyłącznie wartości nieujemne, choć nie koniecznie calkowite. Przez $ F_x $ oznaczmy dystrybuantę zmiennej $ T_x $ tzn.

\begin{center}
	\begin{equation}
		F_x(t) = P(T_x <= t) dla t>= 0,
	\end{equation}
\end{center}\\

gdzie P oznacza prawdopodobieństwo. Dalsze rozważania będziemy prwadzić przy staśłym założeniu, że dla kaz∂ego $ x >= 0 $, dystrybuanta $ F_x $ jest funkcją ciągłą.\\

Wprowadzimy teraz oznaczenia które, podobnie jak notacja aktuarialna w dalszej części wykładu, są zgodne z Międzynarodowym Systemem Oznaczeń AKtuarialnych, obowiązujących od 1898 roku. Zgodnie z tym systemem:\\

(i) $ _tq_x $ oznacza pradopodobieństwo śmierci x-latka przed upływem czastu t,\\

\begin{center}
	\begin{equation}
		_tq_x = F_x(t)
	\end{equation}
\end{center}\\

(ii) $ _tp_x $ oznacza pradopodobieństwo zdarzenia polegającego na tym, że x-latek przeżyje więcej niż czas t,\\

\begin{center}
	\begin{equation}
		_tp_x = P(T_x > t) = 1 - F_x(t)
	\end{equation}
\end{center}\\

(iii) $ _{s|t}q_x $ oznacza pradopodobieństwo zdarzenia polegającego na tym, że x-latek przeżyje jeszcze s lat, a nastepnie umrze przed upływem czasu t,

\begin{center}
	\begin{equation}
		_{s|t}q_x = P(s < T_x <= s + t) = F_x(s + t) - F_x(s)
	\end{equation}
\end{center}\\

(iv) $ _tp_{[x] + s} $ oznacza pradopodobieństwo warunkowe przeżycia przez x-latka kolejnych t lat pod warunkiem, że wcześniej przeżyje on s lat,

\begin{center}
	\begin{equation}
		_tp_{[x] + s} = P(T_x > s + t|T_x > s)
	\end{equation}
\end{center}\\

(v) $ _tq_{[x]+s} $ oznacza prowdopodobieństwo, że x-latek umrze przed upływem czasu s + t pod warunkie, że wczesniej przeżyje czas s,

\begin{center}
	\begin{equation}
		_tq_{[x] + s} = P(T_x <= s + t|T_x > s)
	\end{equation}
\end{center}\\

Ponadto, przyjmujemy konwencje, że jeżeli jakiś indeks jest równy 1, to pomijamy go w odpowiednim symbolu, np. zamiast $ _1p_x $ będziemy pisać $ p_x $.\\

Można sprawdzić, że dla dowolnych $ x,s,t >= 0 $ zachodza równości

\begin{center}
	\begin{equation}
		_{s|t}q_x = _sp_x - _{s+t}p_x = _{s+t}q_x - _sq_x
	\end{equation}
\end{center}\\

Zauważmy, że dla dowolnych $ x,t >= 0  $ mamy $ _tp_x + _tq_x = 1 $\\

Niech $ < x > $ oznacza część całkowitą liczby rzeczywistej x. Zmienną losową $ K_x = < T_x > $ nazywamy \textbf{obciętym przyszłym czasem życia} x-latka. Zmienna $ K_x $ przyjmuje wyłącznie wartości całkowite nieujemne i wyraża liczę ukończonych przez x-latka pełnych lat życia. \\

\newpage

\section{Hipoteza agregacji}

W praktyce ubezpieczeniowej przyjmuje się, że dla kazdego $ x \epsilon N_0 $ dany jest rozkład zmiennej $ K_x $. Jednakże budowanie tablic dla każdego $  x \epsilon N_0$ byłoby w pkratyce bardzo kłopotliwe i w istocie nie zawsze sensowne.\\

Wobec tego przyjmuje się tzw. hipotezę agregacji, które pozwala wyrazić każdy z rozkładów $ K_x $ za pomocą rozkładu zmiennej $ K_0 $. Mówimy mianowicie, że rodzina rozkładów $ (K_x : x \epsilon N_0) $ spełnia \textbf{hipotezę agregacji} jeżeli, dla dowolnych $ x,k \epsilon N_0 $ spełniających warunek $ P(K_x >= k) > 0 $, zachodzi równość

\begin{center}
		P(K_x >= k) = P(K_0 >= x + k|K_0 >= x), (HA)
\end{center}\\

Przy założeniu hipotzy agregacji, dla dowolnych $ x >= 0 i k \epsilon N_0 $ zachodzi równość

\begin{center}
	\begin{equation}
		P(K_x = k) = _kP_x\cdot q_{x+k}
	\end{equation}
\end{center}\\

\newpage

\section{Hipotezja jednostajności}

Zakladając hipotezę agregacji możemy wyznaczyć wartośi $ _nP_x $ dla wszystkich $ n, x \epsilon N_0 $, mając dae jedynie wartości $ _nP_0 $. W celu opisania zachowania się funkcji $ t -> t p_x dla x \epsilon N_0 $, między punktami $ t = 0,1.....,$, stosuje się tzw. \textbf{hipoetzy interpolacyjne}. Omówimy jedynie hipotezę jednostajności.\\

Mówimy, że rozkład $ T_x $ spełnia \textbf{hipotezę jednostajności} jeżeli, dla dowolnych $ n \epsilon N_0 $ oraz $ \lambda \epsilon [0,1) $ zachodzi równość\\

\begin{center}
	\begin{equation}
		_{n+\lambda}p_x = (1-\lambda)__np_x + \lambda_{n+_1p_x}, (HU)
	\end{equation}
\end{center}\\

Zauważmy, że (HU) oznacza, że dla każdego $ x \epsilon N_0 $ funckja $ t->_t p_x $ jest iniowa na każdym z przedziałów $ [n, n+1] $, gdzie $ n \epsilon n_0 $//

 \begin{center}
	\begin{equation}
		_{\lambda}p_x = 1 - \lambda q_x
	\end{equation}
\end{center}\\

\begin{center}
	\begin{equation}
		_{\lambda}q_x = \lambda q_x
	\end{equation}
\end{center}\\

Istotnie ustalamy $ x \epsilon N_0 $ i $ \lambda \epsilon [0,1) $. Stosując (HU) z $ n=0 $ otrzymjemy\

$ _{\lambda}p_x = (1 - \lambda) + \lambda p_x = 1 - \lambda (1 - p_x) = 1 - \lambda q_x $\\

co dowodzi (94).\\

\newpage

\section{Tablice trwania życia}

Tablica trwania życia jest powszechnie stosowaną metodą zestawienia informacji dotyczących rozkładu czasu życia. W naszych rozważaniach ograniczymy sie do tablic trwania życia związanych ze zmienną losową $ K_0 $ tzn. z obciętym przyszłym czasem życia 0-latka.\\

Przez \textbf{tablicę trwania życia} (w skrócie TTŻ) zmiennej $ K_0 $ będziemy rozumieć ciąg liczb nieujemnych $ (l_k : k \epsilon N_0) $ spełniających zależność: 

\begin{center}
	\begin{equation}
		P(K_0 >= k) = \frac{l_k}{l_0, dla k \epsilon N_0}
	\end{equation}
\end{center}\\

Liczby $ l_k $ mają następującą interpretację. Liczba $ l_0 $ jest pozcątkową liczebnością generacji (w praktyce najczęściej przyjmuje się $ l_0 = 100000 $). Dla każdego $ k >= l $. $ l_k $ wyraża średnią liczbę członków generacji, dożywających powyżej wieku k. Z definicji TTŻ wynika, że jeżeli $ l_k = 0 $ dla pewnwego $ k \epsilon N_0 $ to dla każdego $ n \epsilon N_0 $ takiego że $ n >= k $, zachodzi $ l_n = 0 $.Zwykle w TTŻ zakłada się istnienie maksymalnego wieku $ \omega = min{k \epsilon N_0 : l_k = 0} $. \\

Okazuje się, że mając daną tablicę trwania życia związaną ze zmienna $ K_0 $ i zakładając hipotezę agregacji, można zbudować tablice trwania życia dla wszystkich $ K_x $, gdzie $ x \epsilon N_0 $. Są to tzw. \textbf{tablice zagregowane}. Opierając się one na równości

\begin{center}
	\begin{equation}
		_kp_x = P(K_x >= k) = \frac{l_x + k}{l_x}, dla k \epsilon N_0
	\end{equation}
\end{center}\\

Z równości (97) wynika, że dla kazdego $ x \epsilon N $ takiego, że $ l_x > 0 $zachodzą równości:

\begin{center}
	\begin{equation}
		p_x = \frac{l_{x+1}}{l_x}
	\end{equation}
\end{center}\\

oraz\\

\begin{center}
	\begin{equation}
		q_x = 1 - p_x = \frac{l_x - l_{x+1}}{l_x}
	\end{equation}
\end{center}\\

Zwykle, poza wartościami $ lx, x = 0,1,...., \omega - 1 $ w tablicach zamieszczone są inne wielkości które można otrzymać za pomoca $ l_x $. Są to np. $ p_x, q_x $ czy też średnia liczba osób z początkowej generacji, które zmarły w wieku x lat, tzn. $ d_x := l_x - l_{x+1} $. \\

\subsection{Przykład 74}

Obliczymy prawdopodobieństwo, że 50-letnia kobieta przeżyje co najmniej 5 lat. tzn $ P(K_{50} >= 5) $\\

wzór (97)\\

$ P(K_{50} >= 5) = _5p_{50} = \frac{l_{55}}{l_{50}}} = 0,9841 $

\subsection{Przykład 75}

Obliczymy prawdopodobieństwo że 85-letni mężczyzna przeżyje co najmniej 0,75 roku.\\

Korzystając z hipotezy jednostajności (94)\\

$ P(T_{85} >= 0,75) = _{0,75}p_{85} = 1 - 0,75 \cdot q_{85} = 0,9145 $

\subsection{Przykład 76}

Obliczymy prawdopodobieństwo że 50-letni mężczyzna przeżyje jeszcze co najmniej 2,5 roku.\\

stosujemy (HU)\\

$ P(T_{50} >= 2,5) = _{2,5}p_{50} = 0,5 \cdot _2p_{50} + 0,5 \cdot _3p_{50} = 0,5\frac{l_52 + l_{53}}{2 \cdot l_{50}} = 0,9815 $

\subsection{Przykład 77}

Obliczymy prawdopodobieństwo, że 20-letnia kobieta przeżyje jeszcze co najmniej 40 lat ale nie więcej niż 45,25 roku.\\

tzn, $ P(40 < T_{20} <= 45,25) $, zatem stosując hipoteżę (HU) i (92) otrzymujemy:\\

$ P(40 < T_{20} <= 45,25) = _{40|5,25}q_{20} = _{40}p_{20} - _{45,25}p_{20} = _{40}p_{20} - ((1 - 0,25)_{45}p_{20} + 0,25 \cdot _{46}p_{20}) = \frac{l_{60}}{l_{20}} - (0,75\frac{l_{65}{l_{20}} + 0,25 \cdot \frac{l_{66}}{l_{20}}) = 0,0410 $\\

\newpage

\section{Podstawowe rodzaje ubezpieczeń życiowych}

\subsection{Ubezpieczenie na życie} - przedmiotem jest śmierć ubezpieczonego w trakcie trwania okresu ubezpieczenia.. Typowym świadczeniem dodatkowym jest objęcie ochroną ubezpieczeniową utraty zdolności do pracy lub wypłata podwyższonej sumy ubezpieczenia, gdy smierć nastapiła wskutek wypadku.\\

Celem tego ubezpieczenia jest:

\begin{itemize}
	\item zrekopensowanie utraconych dochodów przez okres umożliwiający rodzinie zmarłego znalezienie nowych źródeł utrzymania;
	\item zapewnienie spłaty zobowiązań, które stanowiłyby nadmierny wydatek dla rodziny zmarłego;
\end{itemize}\\

Wyróżnia się dwa warianty ubezpieczenia na życie:\\

1. Ubezpieczenie terminowe\\

2. Ubezpieczenie na całe życie\\

\textbf{Ubezpieczenie terminowe na życie} może być prowadzone przy stałej sumie ubezpieczenie lub przy obniżającą się sumą ubezpieczenia, równą niespłaconej kwocie kredytu. Składka wpłacana jest jednorazowo lub okresowo, przeważnie w równych ratach. Jest to ubezpieczenie o charakterze typowo ochronnym.\\

\textbf{Ubezpieczenie na całe życie} w kazdym przypadku kończy się wypłatą świadczenie z tytułu śmierci ubezpieczonego. Składka opłacana jest okresowo do końca trwania okresu ubepzieczenia albo do ustalonego czasu (np. do osiąfnięcia przez ubezpieczonego wieku emerytalnego). Ubezpieczenie jest [rowadzone albo przy stałej składce, albo przy wzrastającej skłądce (i wzrastającej odpowiednio sumie ubezpieczenia) w celu zneutralizowania negatywnego wpływu inflacji i utrzymania realnej wartości sumy ubezpieczenia na stałym poziomie. W zależności od zapisów umowy, po ustalonym okresie trwania umowy ubezpiezcający ma prawo zaprzestać ubezpieczenia i odzyskać część składki. Ubezpieczenie może być również zamienione na bezskładkowe.



\end{document}