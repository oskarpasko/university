\documentclass[12pt, letterpaper]{article}
\usepackage{graphicx}
\usepackage[T1]{fontenc}
\usepackage[polish]{babel}
\usepackage[utf8]{inputenc}
\usepackage{times}


\graphicspath{{images/}}

%--------------------------------------------------------------------------------------------------
%       TITLE SECTION
%--------------------------------------------------------------------------------------------------
\begin{titlepage}

\begin{center}
	\includegraphics[scale=0.2]{ur_inf_logo}\\ \\ \\ \\
\end{center}

\begin{center}

	{ \huge \bfseries Ochrona własności intelektualnej \\ i przemysłowej}\\ \\[0.4cm] 

	\textsc{\Large Patenty: Innowacje, Ochrona i Rozwój Gospodarczy}\\[0.5cm] \\ \\ \\ \\ 
	
	\vspace{0.8cm}	
	
	\emph{Autor:} \\
	\textbf{Oskar Paśko} (117987)\\
	
	
	\vspace{0.8cm}
	
	\emph{Kierunek:} \\
	Informatyka i ekonometria
	
	\vspace{8cm}
	
	\emph{Prowadzący:} \\
	dr Michał Chajda\\ \\ \\ \\ 
	
	\vspace{2cm}
	
	Rzeszów, 2023
\end{center}
\end{titlepage}

\usepackage{geometry}
 \geometry{
 a4paper,
 total={170mm,257mm},
 left=25mm,
 top=25mm,
 right=25mm,
 bottom=25mm 
 }
 
%--------------------------------------------------------------------------------------------------
%       BEGIN DOCUMENT
%--------------------------------------------------------------------------------------------------

\begin{document}
\newpage

\section{Wprowadzenie}

Patenty odgrywają kluczową rolę w dzisiejszym świecie, stanowiąc nie tylko środek ochrony dla wynalazców, ale również bodziec dla innowacji i rozwoju gospodarczego. W niniejszym referacie skoncentrujemy się na znaczeniu patentów, ich wpływie na społeczeństwo i gospodarkę oraz na ewolucji tego instrumentu ochrony intelektualnej.

\section{Definicja i Cele Patentów}

Patent to wyłączne prawo do korzystania z wynalazku na określonym terytorium przez określony czas. Jego celem jest zachęcanie do tworzenia nowych rozwiązań poprzez zapewnienie wynalazcy czasowego monopolu na korzystanie z jego pomysłu. Patenty dzielą się na kilka kategorii, obejmujące wynalazki przemysłowe, wzory użytkowe czy metody przedsiębiorcze.

\section{Rola Patentów w Innowacjach i Rozwoju Gospodarczym}

\subsection{Stymulowanie Innowacji}

Patenty są silnym bodźcem dla innowacji. Wynalazcy, wiedząc, że ich prace są chronione, są bardziej skłonni dzielić się swoimi pomysłami z resztą społeczeństwa, co prowadzi do dalszego rozwoju technologicznego.

\subsection{Wzrost Konkurencyjności}

Ochrona patentowa zwiększa konkurencyjność na rynku. Firmy chętniej inwestują w badania \\ i rozwój, aby zdobyć przewagę konkurencyjną i zyskać ekskluzywne prawa do nowych technologii.

\subsection{Transfer Technologii}

Patenty umożliwiają przedsiębiorstwom przekazywanie technologii, co sprzyja globalnemu rozwojowi gospodarczemu poprzez efektywniejsze wykorzystanie zdobyczy technologicznych.

\section{Wyzwania i Krytyka Systemu Patentowego}

\subsection{Zatory w Systemie Patentowym}

Wzrost liczby zgłoszeń patentowych może prowadzić do zatorów w systemie patentowym, wydłużając czas oczekiwania na przyznanie patentu.

\subsection{Patent Trolli}

Zjawisko tzw. "patent trollingu" polega na wykorzystywaniu patentów nie w celu prowadzenia działalności gospodarczej, ale do dochodzenia roszczeń na drodze sądowej, co utrudnia rozwój innowacyjny.

\subsection{Monopolizacja Rynku}

Niektórzy krytykują system patentowy za to, że umożliwia firmom utrzymanie monopolu na pewne technologie, co może ograniczać zdolność innych do korzystania z tych rozwiązań.

\section{Perspektywy na Przyszłość}

\subsection{Patenty a Globalne Wyzwania}

W obliczu globalnych wyzwań, takich jak zmiany klimatyczne czy zdrowie publiczne, patenty mogą stać się narzędziem do promowania innowacyjnych rozwiązań dla tych problemów.

\subsection{Rozwój Technologii i Dostępność Patentów}

Zmiany w dziedzinie technologii, takie jak sztuczna inteligencja czy biotechnologia, mogą stworzyć nowe wyzwania dla systemu patentowego, wymagając dostosowań i nowych regulacji.

\section{Podsumowanie}

Patenty odgrywają kluczową rolę w wspieraniu innowacji i rozwoju gospodarczego. Jednakże, system ten nie jest pozbawiony wyzwań, takich jak zatory w systemie czy zjawisko patent trollingu. W kontekście dynamicznego rozwoju technologii i globalnych wyzwań, istnieje potrzeba ciągłego dostosowywania i ulepszania systemu patentowego, aby efektywnie wspierał rozwój społeczeństwa. Warto również zastanowić się nad równowagą między ochroną wynalazców \\ a zapewnieniem powszechnego dostępu do innowacyjnych rozwiązań.

\end{document}